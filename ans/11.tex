\documentclass[__main__.tex]{subfiles}

\begin{document}

\qtitle{11}
Явные итерационные методы, понятие устойчивости сходящегося метода простой итерации.\\

\textbf{Явные итерационные методы}\\

Пусть $X = (X;\rho)$ - полное метрическое пространство в котором задана последовательность отображений $\phi_{(\cdot)} = (\phi_k:X^P \to X)$.\\
Явный P-шаговый итерационный метод:\\
Последовательность $\phi_{(\cdot)}$ и НУ $x_0,x_1...x_{p-1} \in X$ определяют явный P-шаговый итерационный метод $Itr(\phi_1;x_0,...,x_{p-1})$ с рабочей формулой:\\
\begin{gather}
\llabel{11:1}
x_n = \phi_k(x_{k-p},x_{k-p+1},...,x_{k-1}) \ \ k \in N \ \ k\geq p
\end{gather}
Согласно формуле \lref{11:1} последовательность $x_{(\cdot)} = (x_k)_N$ называется итерационной последовательностью метода $Itr(\phi_{(\cdot)};x_0,x_1,...,x_{p-1})$. Если $\exists \lim_{k \to \infty} x_k > x_*$. В этом случае, если для $\alpha \geq 1$ и $k \geq 1$ выполняется условие $\rho(x_{k+1},x_*) \leq C \rho^\alpha (x_k,x_*) \ (C>0)$, то число $\alpha$ называют порядком сходимости метода. ($\alpha = 1$ - линейный, $1<\alpha<2$ - сверхлинейный, $\alpha = 2$ -квадратичный).
\end{document}