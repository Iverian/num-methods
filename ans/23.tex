\documentclass[__main__.tex]{subfiles}

\begin{document}

\qtitle{23}
Схема интерполирования Лагранжа гладкой на отрезке функции, понятия её сходимости, аналитической корректности, устойчивости и корректности. Формулировка теоремы Чебышёва об аналитической корректности задачи интерполирования Лагранжа со схемой чебышёвских сеток.\\

\begin{definition}
Пусть $A_{(\cdot)} = (A_k)_{\mathbb{N}}$ - схема равномерных сеток на $[a; b]$ и $|A_k| = k + 1$ для  $k \in \mathbb{N}$ и $f \in X$ . Пусть $L_k = L_k(A; \hat{A} = {}^{>}f_{(k)})$ - $A_k$ - интерполяционный многочлен Лагранжа для $k \in \mathbb{N}$

Тогда $(L_k)_{\mathbb{N}}$ - решение задачи $A$ - интерполирования функции $f$ на $[a; b]$. 

\end{definition}

\begin{definition}
Пусть $g_{(\cdot)} = (g_k)_{\mathbb{N}}$ - решение задачи $A_{(\cdot)}$ - интерполирования функции $f \in x$.
\end{definition}

\begin{definition}
Решение $g_{(\cdot)}$ называют сходящимся (в норме $||\cdot||$ ) если $\exists \lim_{k \to +\infty} g_k = g$ в пространстве кусочно-непрерывных функций на $[a; b]$.
\end{definition}

\begin{definition}
Решение  $g_{(\cdot)}$  называют аналитически корректным, если оно является сходящимся и $\lim_{k \to +\infty} = f$
\end{definition}

\begin{definition}
Решение  $g_{(\cdot)}$ называют корректным, если оно аналитически корректно и для $\forall \varepsilon > 0$ и $\forall k \in \mathbb{N}$ выполняются условия:

$$||\hat{g}_k - f|| < c\varepsilon$$

где $c > 0$ не зависит от $k \in \mathbb{N}$ ${}^{>}\title{f} = \hat{A_k}(f) > \varepsilon_k$, $||{}^{>}\varepsilon_k|| \le \varepsilon$ для $\forall k \in \mathbb{N}$ и $\tilde{g}_{(\cdot)} = (\tilde{g_k})_{\mathbb{N}}$ решение той же задачи $A_{(k)}$ - интерполяции для функции ${}^{>}f_{(k)}$ для $k \in \mathbb{N}$ 
\end{definition}

\begin{definition}
Пусть $\forall k \in \mathbb{N}$ решение $g_k$ и $\tilde{g}_k$ задачи интерполяции функций ${}^{>}f_{(k)}, {}^{>}f_{(k)} \in {}^{>}R^{|A_k|}(A_k)$ для сетки $A_k$ из схемы сеток $A_{(\cdot)} = (A_m)_{\mathbb{N}}$, где $||{}^{>}f_{(k)} - {}^{>}\breve{f}_{(k)}|| < \varepsilon$ и $\varepsilon > 0$ - произвольное число, удолетворяют условию $||g_k - \title{g}_k|| \le c\varepsilon$, где число $c \ge 0$ не зависит от $k$. Тогда рассмотренная задача $A_{(\cdot)}$ - интерполирования схемы сеточных функций $({}^{>}f_{(k)})_{\mathbb{N}}$ называют устойчивой.
\end{definition}

\begin{definition}
Пусть решение задачи $A_{(\cdot)}$ - интерполирования функции $f$ - аналитически корректно и устойчиво. Тогда это решение корректно.
\end{definition}

\begin{definition}
Сетку $A = <r_0, r_1, \text{...}, r_k>$ отрезка $[a, b]$ называют чебышевской, если:

$$r_j = \frac{a+b}{2} + \frac{b-a}{2}\cos\frac{(2j + 1)\pi}{2k + 1}$$ для $j \in \mathbb{Z}_{k + 1}\\ \\ $ ($r_i > r_{i + 1}$, для $i = \overline{0, k - 1}$)  

\end{definition}

\begin{theorem}
Пусть $A_{(\cdot)} = (A_k)_{\mathbb{N}}$ - схема чебышевских сеток $[a; b]$, где $|A_k| = k + 1$ и $L_{(\cdot)} = (L_{k})_{\mathbb{N}}$ - решения задачи $A_{(\cdot)}$ - интерполяции функции $f \in C^{1}([a; b], \mathbb{R})$. Тогда это решение - аналитически корректно.
\end{theorem}

\end{document}