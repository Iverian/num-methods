\documentclass[__main__.tex]{subfiles}

\begin{document}

\qtitle{38}
Понятие аппроксимации линейного многообразия в банаховом пространстве. Табличная аппроксимация и интерпретация на примере сеточно-сплайновой аппроксимации гладких на отрезке функций, согласованная аппроксимация (интерпретация).\\

Пусть $H_{(.)} = (H_k)_N - $ последовательность квази-базисов в пространстве $Y_0,$ что $\forall x_0\in X_0$ существует последовательность $y_{(.)}=<y_k\in [H_k]=Y_k)_N,$ сходящаяся к $X_0$ в пр-ве $Y_0$, то есть $x_0=lim_{k->\infty}y_k.$ Тогда последовательность $H_{(.)}$ называю базой аппроксимирования линейного многообразия $X_0$ \\

Пусть $H_{(.)}=(H_k \in Y_0)_N - $ база аппроксимирования $X_0$ в $Y_0$  и $\forall k \in N$ задана аппроксимация\\
$\hat{p}_k \in hom_c (Y_0,[H_k]=Y_k)$ элементов из $X_0$\\
Тогда последовательность $\hat{p}_{(.)}=(\hat{p}_k)_N $ называют аппроксимированием из элементов из $X_0$. Такое аппроксимирование  $\hat{p}_{(.)}$ называется сходящимся, если $\forall X_0 \exists lim_{R->\infty} \hat{p}_k (x_0) = Y_0$. Такое сходящееся аппроксимирование назовем аналитически-корректной , если $lim_{k->+\infty}\hat{p}_k(x_0)=x_0 \forall x_0\in X_0$\\
Так аналитически-корректное аппроксимирование $\hat{p}_{(.)}$ корректно, если оно устойчиво; то есть $(||\hat{p}_k||)_N$ - ограничена\\

Пример Сеточно-сплайнового аппроксимирования \\
Пусть $Y_0=\underline{M}_B([a;b],R),X_0=\underline{C}([a;b],R) \ ,A_{(.)}=(A_i)_N$ схема устранимо-разрывных сеток $[a;b]$\\
Кроме того, $\hat{H}_{(.)}=(\hat{A}_n)_N$ -- табулирование элементов пространства $X_0$ и \\$B_{(.)}=(\hat{\Psi}_n\in hom_c(^>R^{|A_k|},Spl_{|A_k|}(A_k))$ -- интерпретирование схемы сеточных пространств ($^>R^{|A_k|}(A_k))_N$), где $\hat{\Psi}_k(^>U)=\sum u_i Spl(A_{u_i}{^>C}_i)	$ для $^>u\in ^>R^{|A_k}(A_k)$  и  $k\in N$\\
Тогда $\forall k\in N$ определена аппроксимация $\hat{p}_k=\hat{\Psi}_k \circ \hat{A}_k \in hom_c(Y_0{^>R}^{|A_K|}$ элементов из $X_0$	где $||\hat{p}_k||=1$. Поэтому $\hat{p}_{(.)}=(\hat{p}_k)_N$ -- корректное аппроксимирование элементов из $X_0$  в пространстве $Y_0$\\

Определение табличного аппроксимирования линейного оператора:\\
Пусть $Y_0, Z_0$ -- два банаховых пространства, $\hat{F}$  -- линейный оператор из  $Y_0$  в $Z_0$, $D(\hat{F})=X_0$ и $E(\hat{F})=W_0$ - области определения и значения оператора $\hat{F}$ соответственно.\\
$\hat{\pi}_{(.)}=(\hat{\pi}_K \in hom_c(Y_0,{^>R}^{nk}_*))_N$ и $\hat{\Omega}_{(.)}=(\Omega_k\in (Z_0, {^>R}^m_*$-- табуляции элементов из $X_0 $ и $W_0$\\
Для каждого $k\in N$ определена матрица $F_k\in L(R;n_k,m_k)$  размера  $m_k*n_k$ и соответствующий ей табличный оператор $\hat{F}_k\in hom_c(^>R^n_*,{^>R}^m_k)$
Говорят, что последовательность матриц $F_{(.)}=(\underline{F}(k))_N $ является  $(\hat{\pi}_{(.)},\hat{\Omega}_{(.)}$--табулированием оператора  $\hat{F}$, если $\forall x_0 \in X_0 \ \ ||(\hat{F}_k\circ\hat{\pi}_k\circ \hat{F})(x_0)||\underset{k-> \infty}{\longrightarrow}0$

\end{document}