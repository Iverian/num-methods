\documentclass[__main__.tex]{subfiles}

\begin{document}

\qtitle{45}
Метод Галёркина-Петрова для численного решения линейного уравнения в гильбертовом пространстве, пример численного решения линейного интегрального уравнения Фредгольма II-рода с аналитически заданным непрерывным ядром интегрального оператора этого уравнения.\\

Рассмотрим интегральное уравнение Фредгольма 2-го рода:
\begin{gather}
\varepsilon[y(x)]
\equiv
y(x)-\lambda\int\limits_a^b K(x,t)y(t)dt - f(x) = 0,
\label{45:eq}
\end{gather}
приближенное решение будем искать в виде функций:
\begin{gather}
Y_n(x)=f(x)+\sum_{i=1}^{n}a_i\varphi_i(x),
\label{45:sol}
\end{gather}
с вектором свободных параметров ${^>a}=\left[a_1,\dots,a_n\right>$ и заданной системой ЛНЕЗ функций $\left\{\varphi_i(x)\right\}^n$. Подставляя (\ref{45:sol}) в (\ref{45:eq}), получим невязку:
\begin{gather}
\varepsilon[Y_n(x)]
=
\sum_{i=1}^{n}a_i\left[\varphi_i(x)-\lambda\int\limits_{a}^{b}K(x,t)\varphi_i(t)dt\right] - \lambda\int\limits_{a}^{b}K(x,t)f(t)dt,
\label{45:nev}
\end{gather}
в случае точного решения невязка равна нулю, поэтому параметры ${^>a}$ подбираются так, чтобы минимизировать итоговую невязку, если
\begin{gather}
\lim\limits_{n\rightarrow\infty}Y_n(x)=y(x),
\end{gather}
то взяв достаточно большое число параметров, можно найти решение $y(x)$ с любой наперед заданной точностью.

Согласно методу Галеркина - Петрова, компоненты ${^>a}$ определяются из ортогональности невязки функциям $\varphi_i(x)$:
\begin{gather}
\int\limits_{a}^{b}\varepsilon[Y_n(x)]\varphi_i(x)dx=0
\end{gather}
или из (\ref{45:nev}):
\begin{gather}
\sum_{i=1}^{n}(f_{ij}-\lambda b_{ij})a_j = \lambda c_i
\Longrightarrow
(A-\lambda B){^>a} = \lambda{^>c}
\end{gather}
где
\begin{flalign}
&
f_{ij}
=
\int\limits_{a}^{b}\varphi_i(x)\varphi_j(x)dx,\\
&
b_{ij}=\int\limits_{a}^{b}\int\limits_{a}^{b}K(x,t)\varphi_i(x)\varphi_j(t)dtdx,\\
&
c_i=\int\limits_{a}^{b}\int\limits_{a}^{b}K(x,t)\varphi_i(x)f(t)dtdx,
\end{flalign}

\end{document}
