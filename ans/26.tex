\documentclass[__main__.tex]{subfiles}

\begin{document}

\qtitle{26}
Остаток в форме Коши для задачи интерполяции Лагранжа.
\\
Пусть $A=<r_{0},r_{1}...r_{k}>$ - сетка на $[a,b]$ и f$\in$ $C^{k+1} ([a;b],R)$ и $^>y=\hat{A}(f)=(y_{0},y_{1},..y_{k}\in \mathbb{R}^{n}(A))$ и $Z=Z(A,^{>}y)$ - А-интерполяционный полином Лагранжа для А-сеточной функции $^{>}$.

\begin{definition}
	Полином $\Lambda_{A}(r)=(r-r_{0})(r-r_{1})...(r-r_{k})$, где r- переменная с пробегом в [a,b], называют Ф-сеточным полиномом на [a,b].
	\end{definition}

Функцию R=f-Z$\in$ $C^{k+1} ([a;b],R)$ называют остатком А-интерполяции Лагранжа функции f на [a,b].

	\begin{theorem}
	Теорема об остатке в форме Коши интерполяции Лагранжа.
	
	Для $\forall$ r $\in$ [a,b] остаток R=f-Z А-интерполяции Лагранжа функции f на [a,b] имеет вид :
	\begin{gather}
		\label{ss1}
		R(r)=\frac{f^{k+1}(\xi(r))}{(k+1)!}\Lambda_{A}(r),
	\end{gather}
где $\xi(r)\in$(a,b) - некоторые точки интервала (a,b).
\end{theorem}

\begin{proof}
	Пусть  $r_{a}\in [a;b]$ и $r_{a}\notin A.$ Обозначим $p=\frac{R(r_{k})}{\Lambda_{A}(r_{A})}$ и рассмотрим функцию F=f-Z-p$\Lambda_{A}$. Тогда функция $F\in C^{k+1} ([a;b],R)$ имеет на отрезке [a,b] (k+2) нуля в точках $r_{*}, r_{0},r_{1}...r_{k}\in [a,b]$, где $r_{*}\neq r_{i} \forall i\in \mathbb{Z}_{k+1}.$\\
	 Следовательно, согласно теореме Ролля , функция F имеет на [a,b] (k+1) нуль, функция $F^{n}$- k нулей, $F^{k+1}=f^{k+1}-p(k+1)!$ - нуль на (a;b) в т. $\xi(r)\in(a;b)$. \\
   	Также $\Lambda^{k+1}_{A}=(k+1)!, Z^{k+1}=0$.\\
	 Таким образом  $f^{k+1}(\xi(r))-\frac{R(r_{A})}{\Lambda_{A}(r_{A})}(k+1)!=0.$\\
	  То есть  $R(r_{A}=\frac{f^{k+1}(\xi(r))}{(k+1)!}\Lambda_{A}(r))$
\end{proof}


\end{document}