\documentclass[__main__.tex]{subfiles}

\begin{document}

\qtitle{04}
Число обусловленности квадратной матрицы, овражность симметричной положительно определённой матрицы. Оценка относительной погрешности решения системы линейных алгебраических уравнений (СЛАУ) с квадратной матрицей при заданной относительной погрешности правой части СЛАУ.

\begin{definition}[Число обусловленности квадратной матрицы]\\
	Пусть $A\in GL(R,n)$\\
	Число $cond(A)=\vert\vert A^{-1} \vert\vert *\vert\vert A \vert\vert $ называют числом обусловленности матрицы A.($cond(A)\leq 10$ - A хорошо обусловлена,$cond(A)>100$- плохо обусловлена)\\
\end{definition}
	
\begin{definition}[Овражность симметричной положительно определённой матрицы]\\
	Пусть $A>0$-симметричная матрица и $\vert\vert A \vert\vert= \sqrt{\rho_{spr}(^tAA)^t}=\sqrt{\rho_{spr}(A)^2}=max\lbrace\lambda:\lambda\in Spr(A)\rbrace=\rho_{spr}(A)=\lambda_{max}$.Тогда $A^{-1}>0$-симметричная и $\vert\vert A^{-1} \vert\vert=max\lbrace\frac{1}{\lambda},\lambda \in Spr(A)\rbrace=\frac{1}{\lambda_{min}}, \lambda_{min}=min\lbrace\lambda:\lambda \in spr(A)\rbrace$\\
	Следовательно $cond(A)=\vert\vert A^{-1} \vert\vert *\vert\vert A \vert\vert=\frac{\lambda_{max}}{\lambda_{min}} $- овражность матрицы А.\\
\end{definition}

\begin{definition}[Оценка относительной погрешности решения системы линейных алгебраических уравнений (СЛАУ) с квадратной матрицей при заданной относительной погрешности правой части СЛАУ]\\
	Рассмотрим СЛАУ $A ^>x = ^>b$, $A\in GL(R,n), ^>b\in ^>R^n$- заданы и $^>x=A^{-1}* ^>b$\\
	Вместо этой СЛАУ обычно дана:
	
	$$A* ^>x= ^>b + \triangle^> b$$
	
	где $^>\triangle b \in R^n$ и $\delta=\frac{\vert\vert\triangle^> b\vert\vert}{\vert\vert ^>b\vert\vert}$- относительная погрешность в правой части СЛАУ.\\
	Решение СЛАУ имеет вид:
	
	$$^>x=^>x + \triangle^> x$$
	
	где $A \triangle^> x=  \triangle^> b$, т.е.$\vert\vert A^{-1} \triangle b \vert\vert \leq \vert\vert A^{-1}\vert\vert *\vert\vert \triangle^> b\vert\vert$\\
	Но $^>b=A ^>x$,т.е. $\vert\vert ^>b\vert\vert \leq \vert\vert A\vert\vert*\vert\vert^>x\vert\vert \Rightarrow \vert\vert ^>x\vert\vert \geq \frac{\vert\vert^>b\vert\vert}{\vert\vert A \vert\vert}$\\
	Следовательно, $\frac{\triangle^> x}{^>x}\leq \frac{\vert\vert A^{-1}\vert\vert * \vert\vert \triangle^> b \vert\vert}{\vert\vert ^>b \vert\vert \vert\vert A\vert\vert}=\vert\vert A^{-1}\vert\vert * \vert\vert A \vert\vert * \frac{\vert\vert \triangle^> b\vert\vert}{\vert\vert ^>b \vert\vert}= cond(A)*\frac{\vert\vert \triangle^> b\vert\vert}{\vert\vert ^>b \vert\vert}$\\
	Таким образом 
	
	$$\frac{\vert\vert  \triangle^> x \vert\vert}{\vert\vert ^>x\vert\vert}\leq cond(A)*\frac{\vert\vert \triangle^> b\vert\vert}{\vert\vert ^>b \vert\vert}$$
\end{definition}
\end{document}