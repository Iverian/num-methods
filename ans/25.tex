\documentclass[__main__.tex]{subfiles}

\begin{document}

\qtitle{25}
Задача аппроксимации гладкой на отрезке функции с использованием схемы функций Чебышёва.\\

\begin{theorem}
	Об оптимальном выборе схемы сеток для задачи интерполяции Лагранжа. Для задачи интерполяции Лагранжа на сетке $A=<\tau_{0},\tau_{1},...\tau_{k}\in[a;b]$ в классе всех гладких на данном отрезке функций минимальное уклонение от нуля сеточного полинома $\Lambda_{A}(\tau_{a})=(\tau_{a}-\tau_{0})...(\tau_{a}-\tau_{k})$ будет минимальным, если использовать чебышевскую схему сеток:
	$$A=<\tau_{i}=\frac{a+b}{2}-\frac{b-a}{2}\cos\frac{(2(k-j)+1)\pi}{2(k+1)}:j=0..k>.$$
	Если $f\in C^{k+1}([a,b],\mathbb{R})$, то для остатка $Resr_{A}(\tau)=f(\tau)-L_{k}(\tau) $  (здесь $L_{k}(\tau)$-полином лагранжа k-степени) такой интерполяции Лагранжа функции справедлива оценка :
	$$||Rest_{A}||\leq||f^{k+1}||\frac{1}{(k+1)!2^{k}}. $$
\end{theorem}


\end{document}