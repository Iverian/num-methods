\documentclass[__main__.tex]{subfiles}

\begin{document}

\qtitle{18}
Общие принципы вычисления собственных значений и собственных векторов квадратной матрицы, пример метода Крылова.

Рассмотрим матрицу $A=(a^i_j)^n_m\in L(\mathbb r,n)$. Для определения СЗ сначла надо вычислить её характеристический полином $\mathcal{X}_{rA}$:
\begin{gather*}
	\mathcal{X}_{rA}(\lambda)=Det(A-\lambda E_n)=(-1)^n\lambda^n+p_{n-1}\lambda^{n-1}+\dots+p_1\lambda+p_0,\;где\;p_0=Det(A).
\end{gather*}
Для определения $\mathcal{X}_{rA}$ вычисляется $Det(A)$ и $Det(A-\tau_iE_n)=\mathcal{X}_{rA}(\tau_i)$ для попарно различных $t_i,\;i=\overline{1,n-1}$.\\
Для нахождения коэффициентов $p_1,\dots,p_n$ необходимо решить СЛАУ:
\begin{gather*}
\begin{cases}
	p_1+\tau_1 p_2+\dots+\tau^{n-2}_1 p_{n-1}=\frac{1}{\tau_1}\left(\mathcal{X}_{rA}(\tau_1)-(-1)^n\tau^n_1-Det(A)\right)\\
	\vdots
	p_1+\tau_{n-1} p_2+\dots+\tau^{n-2}_{n-1} p_{n-1}=\frac{1}{\tau_{n-1}}\left(\mathcal{X}_{rA}(\tau_{n-1})-(-1)^n\tau^n_{n-1}-Det(A)\right)
\end{cases}
\end{gather*}
Определитель этой матрицы является определителем Ван-Дер-Монда:
\begin{gather*}
	\begin{vmatrix}
		1 & \tau_1 & \dots & \tau^{n-2}_1 \\
		\vdots & . & . & . \\
		1 & \tau_{n-1} & \dots & \tau^{n-2}_{n-1}
	\end{vmatrix}
=\prod\limits_{1\le i < j\le n-1}(\tau_j-\tau_i)\ne 0\Rightarrow
\end{gather*}
коэффициенты определяются однозначно. Затем можно численно вычислить корни уравнения $	\mathcal{X}_{rA}(\lambda)=0$ - СЗ. Найдя СЗ, можно найти СВ, решив СЛАУ $(A-\lambda E_n)\;^{>}x=\;^{>}0_n.$\\

Пример для метода Крылова.
Пусть дана матрица $A$ и произвольный вектор $\;^{>}x$. Требуется найти полином (коэффициенты полинома).
\begin{gather*}
	A=
	\begin{pmatrix}
		3 & -3 & 1 \\
		3 & -2 & 2 \\
		-1 & 2 & 0
	\end{pmatrix}
	\;^{>}x=
	\begin{pmatrix}
		1 \\ 0 \\ 0
	\end{pmatrix}\Rightarrow\\
	\mathcal{X}_{rA}(\lambda)=
	\begin{vmatrix}
			3-\lambda & -3 & 1 \\
		3 & -2-\lambda & 2 \\
		-1 & 2 & 0-\lambda
	\end{vmatrix}=
	-\lambda^3+\lambda^2-2 \;(p_0=-2, p_1=0, p_2=1)
\end{gather*}
Для проверки воспользуемся методом Крылова:
\begin{gather*}
	\;^{>}x_1=A\;^{>}x=
	\begin{pmatrix}
		3 \\ 3 \\ -1
	\end{pmatrix}
\;	\;^{>}x_2=A\;^{>}x_1=
	\begin{pmatrix}
		-1 \\ 1 \\ 3
	\end{pmatrix}
\;	\;^{>}x_3=
	\begin{pmatrix}
		-3 \\ 1 \\ 3
	\end{pmatrix}\Rightarrow\\
\underset{<\;^{>}x,\;^{>}x_1,\;^{>}x_2]}{\begin{pmatrix}
		1 & 3 & -1 \\
		0 & 3 & 1 \\
		0 & -1 & 3
	\end{pmatrix}}
	\begin{pmatrix}
		p_0 \\ p_1 \\ p_2
	\end{pmatrix}=\;^{>}x_3\Rightarrow
	\;^{>}p=
	\begin{pmatrix}
		-2 \\ 0 \\ 1
	\end{pmatrix}
\end{gather*}
\end{document}
