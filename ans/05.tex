\documentclass[__main__.tex]{subfiles}

\begin{document}

\qtitle{05}
Степенные матричные ряды и аналитические матричные функции.\\

Рассмотрим банахово пространство $(X,\Vert\cdot\Vert)$ с дополнительно определенной операцией умножения элементов.

\begin{definition}[Банахова алгебра]
\emph{Банаховой алгеброй} называется банахово пространство над полем $\mathbb{R}$ $(X,\Vert\cdot\Vert)$ с операцией умножения элементов $\circ$, для которой выполняется
\begin{itemize}
\item
существование нейтрального элемента (единицы) $I_{\circ}$:
$
\forall{x\in X}I_{\circ}\circ{x}=x\circ{I}_{\circ}=x
$;
\item
ассоциативность;
\item
$\forall{\alpha,\beta\in\mathbb{R},x,y\in X}\colon(\alpha x)\circ(\beta y) = (\alpha\beta)\circ(x y)$;
\item
$\forall{x,y,z\in{X}}\colon x\circ(y+z)=x\circ{y}+x\circ{z}\wedge(y+z)\circ{x}=y\circ{x}+z\circ{x}$;
\item
$\Vert I_{\circ} \Vert = 1$;
\item
$\forall{x,y\in X}\Vert x\circ{y} \Vert \le \Vert{x}\Vert\cdot\Vert{y}\Vert$;
\end{itemize}
\end{definition}

\emph{
Банахово пространство -- нормированное линейное пространство, полное по метрике, порожденной нормой. Одним из явных примеров банаховых алгебр является линейное пространство матриц с ассоциативной операцией матричного умножения. 
}

Теперь мы можем аналогично определить степенные ряды над банаховой алгеброй:
\begin{gather}
\forall t\in X \colon \sum_{i=0}^{\infty}a_{k}t^{k}=A(t),
\end{gather}
где $\forall{k\in\mathbb{N}}\colon a_k\in\mathbb{R}$. Вспомним определение радиуса сходимости $R$ степенного ряда
\begin{gather}
\overline{\lim\limits_{m\rightarrow\infty}}\frac{|a_{m+1}|}{|a_{m}|}
=
\overline{\lim\limits_{m\rightarrow\infty}}\sqrt[n]{|a_m|}
=
R>0.
\end{gather}
Тогда мы можем определить функции на банаховых пространствах таким образом: рассмотрим функцию $f(x)$ с рядом Тейлора:
\begin{gather}
f(t)=\sum_{k=0}^{\infty}f_{k}t^{k},
\end{gather}
тогда, заменив аргумент функции $f$ на элемент банахова пространства $x\in X$, получим отображение $f\colon X\rightarrow X$:
\begin{gather}
f(x)=\sum_{k=0}^{\infty}f_{k}x^{k},
\end{gather}
Областью определения полученной функции будет радиус сходимости соответствующего ряда Тейлора, а для $\Vert f(x)\Vert$ получим:
\begin{gather}
\forall{x\in X}\colon\Vert{x}\Vert<R\colon\Vert{f(x)}\Vert\le\sum_{k=0}^{\infty}|f_k|\cdot\Vert{x}\Vert^{k},
\end{gather}

\end{document}