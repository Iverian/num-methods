\documentclass[__main__.tex]{subfiles}

\begin{document}

\qtitle{12}
Метод простой итерации и принцип сжимающих отображений.\\

\textbf{Метод простой итерации}\\
Рассмотрим СЛАУ:\\
\begin{gather}
\llabel{12:1}
^>x=G\cdot ^>x+^>f,
\end{gather}
где $G \in \underline{L}(R;n), ||G||<1, ^>f \in ^>\underline{R^n}$.\\
СЛАУ \lref{12:1} представим в виде:
\begin{gather}
(E_n - G)\cdot ^>x=^>f
\end{gather}
Так как $||G||<1$, то
\begin{gather}
\llabel{12:3}
^>x_*=(E_n-G)^{-1}\cdot ^{>}f =(E_n + G+...+G^{k-1}+G^k+...)\cdot ^>f \text{-решение СЛАУ \lref{12:1} }
\end{gather}
Для численного решения СЛАУ \lref{12:1} используют метод простой итерации с начальным условием $^>x_0 \in ^>R^n$ и рабочей формулой:\\
\begin{gather}
\llabel{12:2}
^>x_k = G \cdot ^>x_{k-1} + ^>f, k \in N
\end{gather}
Рабочая формула \lref{12:2} определяет итерационную последовательность $^>x_{(\cdot)} = (^>x_{k-1})_N$ в которой:\\
\begin{gather}
^>x_1 = G \cdot ^>x_0 + ^>f;\\
^>x_2 = G \cdot ^>x_1 + ^>f = G(G\cdot ^>x_0 + ^>f) = G^2 \cdot ^>x_0 + (E_n+G)^>f;\\
...\\
^>x_k = G\cdot ^>x_k + ^>f = G^k \cdot ^>x_0 + (E+G+...+G^{k-1})^>f
\end{gather}
Используя вид уравнения \lref{12:3} уравнение решается $^>x_*$ СЛАУ \lref{12:1} отсюда получим
\begin{gather}
||^>x_* - ^>x_k|| = ||-G^k \cdot ^>x_0 + (G^k+G^{k+1}+...)^>f|| \\ \leq ||G||^k \cdot ||^>x_0|| + ||G||^k (||E_n|| + ||G|| + ||G||^2+ ...)||^>f||
\end{gather}
Следовательно для $k \in N$ получим приближенное решение $^>x_k$ для СЛАУ \lref{12:1} с оценкой погрешности $\epsilon$ по правилу останова:
\begin{gather}
\llabel{12:4}
||G||^k \cdot ||^>x_0|| + \frac{||G||^k}{1-||G||}||^>f|| < \epsilon
\end{gather}
Из \lref{12:4} следует, что $\lim_{k \to \infty} \ ^>x_k = ^>x_*$, так как $||G||<1$.\\
Если $^>x_0 \Longrightarrow O_n$, то согласно \lref{12:4} правило останова примет вид:
\begin{gather}
\frac{||G||^k}{1-||G||}||^>f||<\epsilon
\end{gather}
Поскольку $\lim_{k \to \infty} \ ^>x_k \Longrightarrow x_*$, то рассмотренный нами метод простой итерации для численного решения СЛАУ \lref{12:1} является сходящимся и аналитически характерным.\\
\textbf{Принцип сжимающих отображений}\\
Рассмотри отображение $\phi: X \to X$ где $X = (\underline{x},\rho)$ - полное метрическое пространство.\\
\begin{definition}
Отображение $X \to X$ называют сжимающим, если есть такое число $q \in [0;1)$, что для $\forall x,y \in X$ справедливо неравенство $\rho (\phi(x),\phi(y)) \leq q\rho(x,y)$. Такое число $q \in [0;1)$ называют коэффициентом сжимания отображения $\phi$.
\end{definition}
Пусть $\phi: X \to X$ - сжимающее отображение с параметром сжимания $q \in [0;1)$.\\
\begin{gather}
\llabel{12:6}
\begin{cases}
x = \phi(x)\\
x \in X
\end{cases}
\end{gather}
Для решения этого уравнения используется метод простой итерации $itr(\phi;x_0)$ с НУ $x_0 \in X$:
\begin{gather}
x_k = \phi(x_{k-1}),k \in N
\end{gather}
Сжимающее отображение $\phi: x \to x$ -непрерывно, то есть $\phi \in C(X,X)$ и имеет единстенную неподвижную точку $x_* \in X$ к которой сходится метод простой итерации $itr(\phi,x_0)$, то есть $x = \phi(x_*)$ -единственное решение уравнения \lref{12:1} и метод простой итерации является корректным для решения уравнения \lref{12:6}.\\
\begin{proof}
Рассмотрим элементы $x_m,x_{m+k} \in X$ -итерационная последовательность $x_{(\cdot)} = (x_n \in X)_N$ метода простой итерации $itr(\phi;x_0)$, где $m,k \in N$\\
Тогда согласно неравенству дэльта-ка для метрики $\rho$ получаем:
\begin{gather}
\rho(x_m,x_{m+k}) = \rho(\phi(x_{m-1})\phi(x_{m+k-1})) \leq q\rho(x_{m-1},x_{m+k-1}) = \\
\rho(\phi(x_{m-2}),\phi(x_{m+k-2})) \leq q^2\rho(x_{m-2},x_{m-2+k}) \leq q^m\rho(x_0,x_k) \leq \\ q^m (\rho(x_0,x_1)+\rho(x_1,x_2)+...+\rho(x_{k-1},x_k)) \leq q^m(\rho(x_0,x_1)+q\rho(x_0,x_1)+...+q^{k-1}\rho(x_0,x1)) \\ \leq q^m(1+q+...+q^{k-1})\rho(x_0,x_1) \leq \frac{q^m}{1-q}\rho(x_0,\phi(x_0))
\end{gather}
Поскольку $0\leq q <1$ отсюда следует, что последовательность $x_{(\cdot)} = (x_n)_N$ -фундаментальная и в силу полноты пространства $X$, последовательность сходится к некоторой точке $x_* \in X$, то есть $\lim_{n \to \infty} x_n = x_*$.\\
Из условия сжимания следует, что $\phi \in C(X,X), \rho(\phi(x),\phi(y)) \leq q\rho(x,y)$.\\
Следовательно справедливо равенства:
\begin{gather}
x_* = \lim_{n \to \infty} \phi(x_n) = \phi(\lim_{n \to \infty}x_n) = \phi(x_*)
\end{gather}
то есть $x_*$ - неподвижная точка отображения $\phi$ является решением уравнения \lref{12:6}.\\
Пусть $y_* \in X, y_* = \phi(y_*)$, тогда:
\begin{gather}
\rho(\phi(x_*),\phi(y_*)) = \rho(x_*,y_*) \leq q\rho(x_*,y_*)
\end{gather}
где $0 \leq q < 1$, то есть $\rho(x_*,y_*) = 0$ и $x_* = y_*$
\end{proof}
Из этого следует, то, что метод простой итерации $itr(\phi,^>x_0)$, где $^>x \in X$, является аналитически корректным для решения уравнения \lref{12:1}
\begin{gather}
\begin{cases}
^>x = \phi(^>x)\\
^>x \in X
\end{cases}
\end{gather}
\textbf{Присоска для понимания}\\
Суть метода простой итерации для решения СЛАУ заключается в том, что мы систему
\begin{gather}
A^>x = ^>b
\end{gather}
приводим к виду
\begin{gather}
^>x = G\cdot ^>x + ^>f
\end{gather}
и ее решение находится как предел последовательности
\begin{gather}
^>x^{(n+1)} = G\cdot ^>x^n + ^>f
\end{gather}
\end{document}