\documentclass[__main__.tex]{subfiles}

\begin{document}
\qtitle{37} Понятия табуляции и табулирования линейного многообразия в банаховом
пространстве. Сеточное табулирование для пространства непрерывных на отрезке
функций. Табуляция в гильбертовом пространстве.\\

\begin{definition}
Последовательность $\hat{\pi}_{(.)}=(\hat{\pi}_k)_N$ назовем табулированем элементов линейного многообразия $X_0$, если $lim_{k->\infty}||\hat{\pi}_k(x_0)||_*=||x_0|| \forall x_0\in X_0$ 
\end{definition}

Пример 1\\
Пусть $A_{(.)}=(A_k)_N$ - схема центрально равномерных сеток $[a;b]$\\
$$Y_0=M_B([a;b],R)\text{ и } X_0=C^1([a;b],R)\subset_M^a \phi_0$$
$\forall k$ положим, что табуляция $\hat{\pi}_k=\hat{A}_k-A_k -$ сеточная табуляция $\Rightarrow \hat{\pi}_k=\hat{A}_k \in hom (Y_0>R^{|A_k|}$ и  $||\hat{A}_k||=1$\\
Тогда $\hat{A}_{(.)}=(\hat{A}_k)_N$ --  табулирование элементов из пространства $X_0$\\
Пусть $Y_0=L_2([-\pi,\pi],R)$ -- гильбертово пространство и $X_0=C_2([-\pi,\pi],R)$ и $h_{(.)}=(h_{n-1})_N$ -- ортонормированный базис $Y_0$\\

Пример 2\\
Если $y_0\in Y_0 $ и  $y_0=\sum_{i=0}<y_0,h_i>k_i=\sum_{i=0}k_ih_i$-ряд фурье $y_0$ для базиса $h_{(.)}$, то для $k\in N$ положим, что $\hat{\pi}_k(y_0)=[u_0,...,u_k>\in ^>E^{k+1}=(^>R^{k+1};||.||_e) $ то есть $\hat{\pi}_k\in hom_c (Y_0,^>E^{k+1})-$ эпиморфизм и $||\hat{\pi}_k||=1.$ Тогда $\hat{\pi}_{(.)}=(\hat{\pi}_k)_N -$ табулирование элементов из $X_0$, так как $||x_0||^2_e=<x_0,x_0>_e=\sum_{i=0}u_i^2=lim_{n->\infty}(u_0^2+u_1^2+...+u_k^2$
\end{document}