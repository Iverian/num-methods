\documentclass[__main__.tex]{subfiles}

\begin{document}

\qtitle{42}
Аппроксимация и (корректное) аппроксимирование линейного оператора в банаховых пространствах, приближённое вычисление значений линейного оператора. Пример вычисления приближённого значения интегрального оператора.

\begin{definition}
	Эпиморфизм есть отображение $m: A \rightarrow B$, такое что если $f*m = h*m$, то $f=h$. 
\end{definition}

\begin{definition}
Пусть $H = <h_0, h_1, ... h_{k-1}>$ - линейно независимая система элементов в $Y_0$, называемая далее квазибазисом и $[H] = [h_0, h_1, ... h_{k-1}] \in Y_0$ - линейная оболочка H. Эпиморфизм $\hat{p} \in Hom_{c}(Y_0, [H])$ называют аппроксимацией элементов $X_0 в Y_0$. Если $x_0 \in X_0$, то $\hat{p}(x_0) \in Y_0$ - аппрокисмация $x_0$ в $Y_0$. 
\end{definition}

Пусть для каждого $k \in N$ в банаховом пространстве $Y_0 = (Y_0, ||.||)$ задан квазибазис $H_k$, индуцирующий линейную оболочку $[H_k] = Y_k$.
\begin{definition}
 Последовательность $H_{(.)} = (H_k)_N$ называют базой аппроксимирования сеперабельного линейного многообразия $X_0 \in Y_0$, если для любого $x_0 \in X_0$ существует такая последовательность $y_{(.)} = (y_k \in Y_k = [H_k])$, что $\lim_{k \rightarrow \infty} y_k = x_0$. Пусть $H_{(.)} = (H_k)_N$ - база аппроксимирования $X_0 \in Y_0$ и для каждого $k \in N$ определена аппр-ция $\hat{p} \in Hom_{c}(Y_0, [H_k] = Y_k)$ элементов $X_0$ в $Y_0$. Тогда последовательность $\hat{p}_{(.)} = (\hat{p}_k)$ называют аппроксимированием $X_0$ в $Y_0$.
\end{definition}
\begin{definition}
Пусть $\hat{p}_{(.)} = (\hat{p}_{k} \in Hom_{c}(Y_0, Y_k))_N$ - аппроксимирование $X_0$ в $Y_0$. Его называют сходящимся, если существует $\lim_{k \rightarrow \infty}\hat{p}_{k}(x_0) \in Y_0$ для любого $x_0 \in X_0$.
\end{definition}
\begin{definition}
 Сходящееся аппроксимирование называют аналитически корректным, если $\lim_{k \rightarrow \infty}\hat{p}_{k}(x_0) = x_0$ для любого $x_0 \in X_0$.
\end{definition}
\begin{definition}
	Аналитически корректное аппроксимирование называют устойчивым, если последовательность $||\hat{p}_{k}||_N$ ограничена.
\end{definition}
\begin{definition}
	Аналитически корректное и устойчивое аппроксимирование называют корректным.
	$(||\hat{p}_{k}(x_0 + \Delta x_0)|| < ||\hat{p}_{k}(x_0)|| + ||\hat{p}_{k}(\Delta x_0)|| < ||\hat{p}_{k}||(||x_0|| + ||\Delta x_0)||, k \in N)$
\end{definition}
\begin{definition}
	Пусть $W_0, Z_0$ - банаховы пространства, $\hat{F}$ - линейный оператор, $G_{(.)} = (G_k)_N$ - база аппроксимирования $W_0$ в $Z_0$, $\hat{P}_{(.)} = (\hat{P}_{k} \in Hom_{c}(Z_0, [G_k] = Z_k))_N$ - корректное аппроксимирование $W_0$ в $Z_0$.
	Последовательность $\hat{F}_{(.)} = (\hat{F}_{k} \in Hom_{c}([H_k] = Y_k, Z_0))_N$ называют:
	- конечномерным $(\hat{p}_{(.)},\hat{P}_{(.)})$. аппроксимированием $\hat{F}$,
	- сходящимся, если для любого $x_0 \in X_0 \rightarrow \lim_{k \rightarrow \infty} (\hat{F}_{k} * \hat{p}_{k}(x_0))$,
	- аналитически корректным, если существует $\lim_{k \rightarrow \infty} (\hat{F}_{k} * \hat{p}_{k}(x_0)) = \hat{F}(x_0)$,
	- устойчивым, если последовательность $(||\hat{F_{k}}||_N)$ ограничена.
\end{definition}

\begin{definition}
Аналитически-корректное и устойчивое $(\hat{p}_{(.)},\hat{P}_{(.)})$ - аппроксимирование $\hat{F}_{(.)} = (\hat{F}_{k})_N$ называют корректным.
\end{definition}
\begin{figure}[h]
	\centering
	\includegraphics[width=.6\linewidth]{42}
	\caption{ }
	\llabel{fig:2}
\end{figure}
\end{document}