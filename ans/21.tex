\documentclass[__main__.tex]{subfiles}

\begin{document}

\qtitle{21}
Понятия сетки, схемы сеток, сеточной функции и схемы сеточных функций. Сеточные отображения для функции определённой на заданном отрезке. Общая постановка задачи интерполяции сеточной функции.\\

\textbf{ Понятия сетки, схемы сеток, сеточной функции и схемы сеточных функций. }

\begin{definition}
Пусть $<\tau_0, \tau_1, \cdots, \tau_k>\; \subset \mathbb{R}$ - список попарно различных точек на $[a; b]$, где $\tau_0 < \tau_1 < \cdots < \tau_k$. Тогда этот список $A =  <\tau_0, \tau_1, \cdots, \tau_k>$ называется \textbf{сеткой} $[a; b]$ и число $stp(A) = \max\{\tau_0 - a; \tau_1 - \tau_0; \cdots; \tau_k - \tau_{k - 1}; b - \tau_k\}$ - называется \textbf{шагом сетки}.
\end{definition} 

\begin{definition}
Если $a = \tau_0, b = \tau_k$ и $\tau_i - \tau_{i - 1} = \frac{b - a}{k}$ для $i = \overline{1, k}$, то сетка $A$ называется \textbf{равномерной}, в этом случае $stp(A) = \frac{b - a}{k}$
\end{definition}

\begin{definition}
Пусть $A$ - равномерная сетка $[a; b]$ и $B = \left< \theta_1 = \frac{\tau_0 + \tau_1}{2}; \theta_2 = \frac{\tau_1 + \tau_2}{2}; \cdots; \theta_k = \frac{\tau_{k - 1} + \tau_{k}}{2} \right>$. Тогда сетку $B$ называют \textbf{центрально равномерной сеткой}, её шаг равен $stp(B) = \frac{b - a}{k}$
\end{definition}

\begin{definition}
Точки $\tau_0, \tau_1, \cdots, \tau_k$ сетки $A$ называются \textbf{узлами сетки}
\end{definition}

\begin{definition}
Пусть для каждого $k \in \mathbb{N}$ задана сетка $A_k$ отрезка $[a; b]$ и $\lim_{k \to +\inf} stp(A_k) = 0$. Тогда последовательность $A_{(\cdot)} = (A_k)_{\mathbb{N}}$ называется \textbf{схемой сеток} $[a; b]$.
\end{definition}

\begin{definition}
Пусть $A = <\tau_0, \tau_1, \cdots, \tau_k>$ - сетка $[a; b]$ и задано отображение $\hat{A}: A \to \mathbb{R}$ Тогда говорят, что задана А - \textbf{сеточная функция} ${}^{>}y \in {}^{>} \mathbb{R}^{|A|}(A)$, для которой ${}^{>}y = [y_0, y_1, \cdots, y_k>$ и $y_j = \hat{A}(\tau_j)$ для $j \in \mathbb{Z}_{k+1} = \{0, 1, 2, \cdots, k\}.$ 
\end{definition}

\begin{definition}
Пусть $A_{(\cdot)} = (A_k)_{\mathbb{N}}$ - схема сеток $[a; b]$ и для каждого $k \in \mathbb{N}$ определена $A_k$ - сеточная функция ${}^{>}y \in {}^{>} \mathbb{R}^{|A|}(A)$. Тогда последовательность ${}^{>}y^{(\cdot)} = ({}^{>}y^{(k)})$ называется $A_{(\cdot)}$ - \textbf{схемой сеточных функций}.
\end{definition}

\begin{definition}
Пусть $f : [a; b] \to \mathbb{R}$ - функция, определенная на $[a; b]$ и $A$ - сетка на $[a; b]$. Кроме того, $M([a; b], \mathbb{R})$ - линейное пространство $\mathbb{R}$ функций, определенных на $[a; b]$ (отметим, что $\mathbb{M}([a; b]; \mathbb{R})$ - банахово пространство. Тогда определено $A$ - сеточное отображение: $\hat{A}: \mathbb{M}([a; b]); \mathbb{R}) to {}^{>}\mathbb{R}^{|A|}(A)$, для которого $\hat{A}(f) = [f(\tau_0), f(\tau_1), \cdots, f(\tau_n)>$, если $A = <\tau_0, \tau_1, \cdots, \tau_k>$
\end{definition}

\begin{definition}
Пусть $A_(\cdot) = (A_k)_{\mathbb{N}}$ - схемы сеток $[a, b]$. Тогда, $\hat{A}_{(\cdot)} = (\hat{A}_ k (f) {}_{\mathbb{N}}$ - $A_{(\cdot)}$ схема сеточных функций для функций $f \in \mathbb{M}([a; b], \mathbb{R})$.
\end{definition}

\textbf{Постановка задачи интерполяции для сеточных функций}

Пусть ${}^{>}f \in {}^{>}\mathbb{R}^{|A|}(A)$ - $A$ сеточная функция на $[a; b]$, где $A = <\tau_0, \tau_1, \cdots, \tau_k>$ - сетка на $[a; b]$.\\

Постановка задачи интерполяции $A$-сеточных функций ${}^{>}f = [f_0, f_1, \cdots\, f_k>$ формулируется следующим образом:\\

Найти такую конструктивную функцию $y \in \mathbb{M}([a; b], \mathbb{R})$ для которой $\hat{A}(g) = [g(\tau_0), g(\tau_1), \cdots, g(\tau_k)> = {}^{>}f$, т.е. доопределить $A$-сеточную функцию ${}^{>}f$ в межузельном пространстве так, чтобы доопределяемая функция $g: [a, b] \to \mathbb{R}$ совпадало по значениям в узлах сетки $A$ с соответствующими значениями $A$-сеточной функции ${}^{>}f$


\end{document}