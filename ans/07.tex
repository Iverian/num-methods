\documentclass[__main__.tex]{subfiles}

\begin{document}

\qtitle{07}
Понятие стабилизирующего функционала и стабилизированный МНК для решения СЛАУ.


\begin{definition}
	Рассмотрим СЛАУ $b^{>} = A * x^{>}$, запишем для нее нормальную СЛАУ  $A^{T}b^{>} = A^{T}A*x^{>}$
Пусть $H \in GL(R,n)$ является симметричной положительно определенной матрицей (H > 0), т.е. $||x^{>}||^2 = x^{<}*H*x^{>}$ - квадрат H-длины вектора $x^{>} \in E^{>,k}$, определяемого скалярным произведением:
\begin{equation}
\ <x^{>};y^{>}>_H = x^{<} H y^{>},
\llabel{eq:1}
\end{equation}
для $ x^{>},y^{>} \in R^{>,n}$ и $||x^{>}||_H$ - Н-длина вектора  $x^{>}$. Квадратичный функционал $||.||_{H}^{2} : E^{>,k} \rightarrow R$ называют стаблизирующим для нашей нормальной СЛАУ. А вектор $x^{>} = Пр_H (O^{>}_{n}; Sol(A^{T}A; A^{T}b^{>}))$. Другими словами, $x^{>}$ - вектор из $R^{n}$, на котором достигается минимальное значение стабилизатора $||.||_{H}^{2}$ при условии, что  $A^{T}b^{>} = A^{T}A*x^{>}$.
\end{definition}
\begin{figure}[h]
	\centering
	\includegraphics[width=.6\linewidth]{7}
	\caption{ }
	\llabel{fig:2}
\end{figure}
\end{document}