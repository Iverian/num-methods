\documentclass[__main__.tex]{subfiles}

\begin{document}

\qtitle{41}
Табуляция и (корректное) табулирование линейного оператора в банаховых пространствах, приближённое вычисление значения линейного оператора. Пример вычисления приближённого значения интегрального оператора.

\begin{definition}
	Эпиморфизм есть отображение $m: A \rightarrow B$, такое что если $f*m = h*m$, то $f=h$. 
\end{definition}

\begin{definition}
	Банахово пространство — нормированное векторное пространство, полное по метрике, порождённой нормой.
\end{definition}
Рассмотрим в банаховом пространстве $Y_0$ линейное многообразие $X_0 \in Y_0$ и конечномерное нормируемое пространство $R^{> n}$
\begin{definition}
	Эпиморфизм $\hat{\pi} \in Hom_{c}(Y_0, R^{> n})$ называют табуляцией $X_0$, если $||\hat{\pi}||=1$
\end{definition}
Пусть $X_{0}$ - линейное многообразие в банаховом пространстве $Y_{0} = M_{B}([a,b],R)$ и $A_{(.)} = (A_{k})_{N}$ - схема сеток $[a,b]$, индуцирующая для каждого $k \in N$ табуляцию $\hat{A_{k}} \in Hom_{c}(Y_{0}, R^{>|A|}(A))$ линейного многообразия $X_{0}$ в пространстве $Y_{0} = (Y_{0},||.||)$ Поскольку 
 $\hat{A_{k}}$ - табуляция, то $|| \hat{A_{k}}||$ = 1 и  $\hat{A_{k}}$ - эпиморфизм.

\begin{definition}
	Последовательность $\hat{A_{(.)}} = (\hat{A_{k}})_{N}$ называют табулированием $X_{0}$, если для любого $x_0 \in X_0 \rightarrow \lim_{k \rightarrow \infty} ||\hat{A_{k}}(x_0)|| =||x_0||$.
\end{definition}

\begin{definition}
	Табулирование $\hat{A_{(.)}}$ можно назвать схемой табуляции $X_{0}$. Это табулирование устойчиво, так как $||\hat{A_{k}}|| = 1$, для любого $k \in N$, и аналитически корректно, т.е. $\lim_{k \rightarrow \infty} ||\hat{A_{k}}(x_0)|| =||x_0||$. Устойчивое и аналитически корректное табулирование называют корректным.
\end{definition}

\end{document}