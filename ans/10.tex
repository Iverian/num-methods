\documentclass[__main__.tex]{subfiles}

\begin{document}

\qtitle{10}
Модель полиномиальной регрессии.\\

Модель полиномиальной регрессии описывает зависимость регрессора $Y \in R$ от значений
одного факторова $t \in R$.Согласно модели полиномиальной регрессии, эта зависимость
имеет вид:
\begin{gather}
\llabel{10:1}
Y = x_*^0+tx_*^1+...+t^kx_*^k+\epsilon
\end{gather}
где $^>x_* = [x_*^0,x_*^1,...,x_*^k> \in ^>E^{k+1}$ - неизвестный вектор параметров тренда $\widetilde{Y}$ модели \lref{10:1}, имеющего вид:
\begin{gather}
\widetilde{Y} = x_*^0+tx_*^1+...+t^kx_*^k
\end{gather}
и $\epsilon \sim N(0,\sigma)$ - случайная состовляющая модели \lref{10:1},являющаяся нормально распределённой случайной величиной с нулевым математическим ожиданием и
неизвестным среднеквадратичным отклонением $\sigma$.\\
Для оценки неизвестных вектора тренда $^>x_*=[x_*^0,x_*^1,...,x_*^k> \in ^>E^{k+1}$ и параметра $\sigma$ случайной состовляющей модели $\epsilon \sim N(0,\sigma)$ линейной регрессии \lref{10:1} проводится эксперимент, в котором измеряются $m>k+1$ значений $y^1,...,y^m \in R$ регрессора модели \lref{10:1} для $m$ попарно различных значений $t_1,...,t_m \in R$ единственного фактора модели \lref{10:1}. После этого рассматривается СЛАУ:
\begin{gather}
\llabel{10:2}
\begin{cases}
x^0+t_1^1x^1+...+t_1^kx^k = y^1 = \widetilde{y}^1 +\epsilon^1;\\
...\\
x^0+t_m^1x^1+...+t_m^kx^k = y^m = \widetilde{y}^m +\epsilon^m;
\end{cases}
\end{gather}
где $\epsilon^1,...,\epsilon^m$ - $m$ независимых реализаций случайной величины $\epsilon \sim N(0,\sigma)$ и $\widetilde{y}^i \in R$ - значение неизвестного тренда модели для значения $t_i \in R$ фактора модели \lref{10:1} $(i= \overline{1,m})$. Введем обозначения:
\begin{gather}
\llabel{10:3}
^>x = [x^0,x^1,...,x^k> = \begin{pmatrix}
x^0\\
\vdots\\
x^k
\end{pmatrix} \in ^>E^{k+1},
\qquad 
^>y=[y^1,...,y^m> = \begin{pmatrix}
y^1\\
\vdots\\
y^m
\end{pmatrix} \in ^>E^m
\end{gather}
\begin{gather}
A = \begin{pmatrix}
1 & t_1 & ... & t_1^k\\
1 & t_2 & ... & t_2^k\\
\vdots & \vdots & ... & \vdots\\
1 & t_m & ... & t_m^k
\end{pmatrix}
\end{gather}
Согласно обозначениям \lref{10:3} СЛАУ \lref{10:2} запишется в виде:\\
\begin{gather}
\llabel{10:4}
A^>x=^>y
\end{gather}
где $A \in L(R;k+1,m)$ - матрица размера $m \times (k+1)$ $m$ -строк, $k+1$ - столбцов и предполагается, что $m>k+1$. Поскольку значения $t_1,...,t_m \in R$ -попарно различны, $rg(A) = k+1$ (в матрице $A$ есть квадратная подматрица размера $(k+1)\times(k+1)$ с
ненулевым определителем Ван-дер-Монда).\\
В качестве оценки неизвестного вектора $^>x_* =[x_*^0,x_*^1,...,x_*^k> \in ^>E^{k+1}$, являющейся МНК-решением СЛАУ \lref{10:4}, то есть решение нормальной для СЛАУ системы:
\begin{gather}
A^T A ^>x = A^T\cdot ^>y
\end{gather}
Кроме того, компонента $\widetilde{Y} (i = \overline{1,m})$ вектора
\begin{gather}
^>\widetilde{Y} = [\widetilde{Y},...,\widetilde{Y}^m> = \begin{pmatrix}
\widetilde{Y}^1\\
\vdots\\
\widetilde{Y}^m
\end{pmatrix}
 = A^>u \in ^>E^m
\end{gather}
представляет оценку значения тренда модели \lref{10:1} для значения $t_i \in R$ единственного ее фактора и величина $\hat{\sigma}^2$, для которой:
\begin{gather}
\hat{\sigma}^2 = \frac{1}{m-k-1} \sum_{i=1}^m (y^i - \widetilde{Y}^i)^2
\end{gather}
является несмещённой и состоятельной оценкой для дисперсии случайной составляющей $\epsilon \sim N(0,\sigma)$ этой модели.\\
\end{document}