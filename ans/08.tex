\documentclass[__main__.tex]{subfiles}

\begin{document}

\qtitle{08}
Модель парной линейной регрессии.

В этой модели зависимость величины y от переменной, называемой фактором, x с пробегом в $R$ имеет вид:
\begin{equation}
 \ y = \alpha_0 + \alpha_1 * x + \epsilon,
\llabel{eq:1}
\end{equation}

где $\epsilon$ - случайная составляющая модели, имеющая стандартный нормальный закон распределения с неизвестной дисперсией, и $\alpha^> = [\alpha_0 , \alpha_1 ]$ - вектор неизвестных параметров тренда модели $y = \alpha_0 + \alpha_1 * x$.

В эксперименте для значений $x_1, x_2, ... x_n$ переменной x модель $(1)$ измеряются соответствующие значения $y_1, y_2, ... y_n$ величины y, называемой регрессором. Значения представлены ниже:
 $$
 \left(
 \begin{matrix}
 i & x & y \\
 1 & x_1 & y_1\\
 2 & x_2 & y_2 \\
 ... & ... & ...\\
 n & x_n & y_n \\
 \end{matrix}
 \right)
 $$

Согласно модели $(1)$ и представленной таблице для оценки вектора $\alpha$ рассматривается приближенная СЛАУ (как правило, несовместная):

\begin{gather}
	\begin{cases}
		\ y_1 = \alpha_0 + \alpha_1 * x_1,\\
		\ y_2 = \alpha_0 + \alpha_1 * x_2,\\
		\ ..... \\
		\ y_n = \alpha_0 + \alpha_1 * x_n,
	\end{cases}
\end{gather}
 
В виде системы это запишется следующим образом:
\begin{gather}
	\label{2}
	X*d^{>} = y^{>}, 
\end{gather}
где 
$X = \left(
\begin{matrix}
1 & x_1\\
1 & x_2\\
... & ...\\
1 & x_n\\
\end{matrix}
\right)$ 

$Y = \left(
\begin{matrix}
y_1\\
y_2\\
...\\
y_n\\
\end{matrix}
\right)$ 

Возможную несовместность СЛАУ (2) иллюстрирует рисунок 1.

\begin{minipage}{.35\linewidth}
	\includegraphics[width=1\linewidth]{8}
\end{minipage}
\hfill
\begin{minipage}{.6\linewidth}
	Построенная на рисунке линия называется трендом и задается уравнением $y = \alpha_0 + \alpha_1 * x$. Для оценки вектора $\alpha^{>}$ решают задачу:
	\begin{gather}
		\label{2}
		g(\alpha^{>}) = \sum_{i=1}^{n}(\alpha_0 + \alpha_1 * x_i - y_i)^2 \rightarrow min, 
	\end{gather}
Решение этой задачи обеспечивает близкую к экспериментальным точкам прямую. Решение этой задачи приводится ниже.
\end{minipage}\\

\begin{gather}
	\begin{cases}
		\frac{\partial y}{\partial  \alpha_0} = 0;\\
		\frac{\partial y}{\partial  \alpha_1} = 0;
	\end{cases}
\end{gather}
Отсюда получаем, что
\begin{gather}
	\begin{cases}
	 \sum_{i=1}^{n}(\alpha_0 + \alpha_1 * x_i - y_i) = 0;\\
	 \sum_{i=1}^{n} x_i (\alpha_0 + \alpha_1 * x_i - y_i) = 0;
	\end{cases}
\end{gather}
Далее:
\begin{gather}
	\begin{cases}
	\sum_{i=1}^{n}(\alpha_1 * x_i) + n*\alpha_0 = \sum_{i=1}^{n}x_i y_i;\\
	\sum_{i=1}^{n}(x_i   \alpha_1 * x_i) + \sum_{i=1}^{n}(x_i * \alpha_0) = \sum_{i=1}^{n} x_i y_i;
\end{cases}
\end{gather}
Подставив значения X и Y, получим $X^{T}*X*\alpha^{>} = X^{T}*y^{>}$. $X^{T}*X$ можно записать матрицей:
 $\left(
\begin{matrix}
n & \sum x_i\\
\sum x_i  & \sum (x_i)^2\\
\end{matrix}
\right)$
Если rang(X) = 2, то $X^{T}*X \in GL(R,2)$. В этом случае МНК-решение СЛАУ 2 ищется как решение СЛАУ 5 в виде 
\begin{equation}
\ (X^{T}*X)^{-1} *X^{T}*y  = B,
\llabel{eq:6}
\end{equation}
где $B = \left(
\begin{matrix}
\hat{\alpha_0} \\
\hat{\alpha_1}\\
\end{matrix}
\right)$
Из вида 4 функции g следует, что $g^{''} = \left(
\begin{matrix}
\frac{(\partial)^2 g}{\partial  (\alpha_0)^2} & \frac{(\partial)^2 g}{\partial(\alpha_0)\partial (\alpha_1)} \\
\frac{(\partial)^2 g}{\partial(\alpha_0)\partial (\alpha_1)} & \frac{(\partial)^2 g}{\partial  (\alpha_1)^2}\\
\end{matrix}
\right)$ обеспечивает минимум функции 4.
Для $n > k+1$ справедлива оценка $\hat{\partial}$:
 \begin{gather}
 	\begin{cases}
 		\hat{\partial^2} = \frac{\sum_{i=1}^{n}\hat{\alpha_0} + \hat{\alpha_1}*x_1^2}{n-2}
 	\end{cases}
 \end{gather}
\end{document}