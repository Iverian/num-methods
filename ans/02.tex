\documentclass[__main__.tex]{subfiles}

\begin{document}

\qtitle{02}
Метод прогонки для СЛАУ с трёх-диагональной матрицей.

\begin{definition}
	Говорят, что матрица $A = (a^i_j)^n_n \in L(\mathbb{R},n)$ имеет диагональное преобразование, если $diag(A) = min\lbrace |a^i_i| - \sum_{j=1, j \neq i}^{n} |a^i_j|: i = \overline{1,n} \rbrace > 0$.		
\end{definition}

\begin{theorem}
	Пусть матрица $A \in L(\mathbb{R},n)$ имеет диагональное преобразование $diag(A) > 0$. Тогда $A \in GL(\mathbb{R},n)$ и $\lVert A^{-1} \rVert \leqslant \frac{1}{diag(A)}$.
\end{theorem}

\begin{proof}
	Рассмотрим совместную СЛАУ
	\begin{equation}
	\llabel{2_1}
	A \cdot \;^{>}x = \;^{>}b.
	\end{equation}
	где $\;^{>}x \in \;^{>}\mathbb{R}^n$ - решение СЛАУ \lref{2_1}, где 
	\begin{equation}
	\llabel{2_2}
	\lVert \;^{>}x \rVert = max \lbrace |x^1|, |x^2|, ... , |x^n| \rbrace = |x^j|, j \in \mathbb{N}_n = <1, 2, ..., n>.
	\end{equation}
	Представим СЛАУ \lref{2_1} в виде:
	\begin{gather}
	\llabel{2_3}
	\begin{cases}
	a^1_1 x^1 = b^1 - \sum_{k=2}^{n} a^1_k x^k \\
	a^2_2 x^2 = b^2 - \sum_{k=1, k\neq 2}^{n} a^2_k x^k \\
	...\\
	a^n_n x^n = b^n - \sum_{k=1}^{n-1} a^n_k x^k	
	\end{cases}
	\end{gather}
	Из \lref{2_3} получаем 
	\begin{gather*}
	\begin{cases}
	|a^1_1| |x^1| \leqslant |b^1| + \sum_{k=2}^{n} |a^1_k| |x^k| \\
	|a^2_2| |x^2| \leqslant |b^2| + \sum_{k=1,k \neq 2}^{n} |a^2_k| |x^k| \\
	... \\
	|a^n_n| |x^n| \leqslant |b^n| + \sum_{k=1}^{n-1} |a^n_k| |x^k|
	\end{cases}
	\end{gather*}
	Преобразуем:
	\begin{gather*}
	\begin{cases}
	|a^1_1| |x^1| - \sum_{k=2}^{n} |a^1_k| |x^k| \leqslant |b^1| \\
	|a^2_2| |x^2| - \sum_{k=1,k \neq 2}^{n} |a^2_k| |x^k| \leqslant |b^2| \\
	... \\
	|a^n_n| |x^n| - \sum_{k=1}^{n-1} |a^n_k| |x^k| \leqslant |b^n|
	\end{cases}
	\end{gather*}
	
	Из \lref{2_2} и $|a^j_j| - \sum_{k=1,k\neq j}^{n} |a^j_k| \geqslant diag(A) \Rightarrow (|a^j_j| - \sum_{k=1,k \neq j}^{n} |a^j_k|) \cdot \lVert \;^{>}x \rVert \leqslant |b^j|$
	Если $\;^{>}b = \;^{>}O_n$, то отсюда следует (т.к. $diag(A)>0$), что $\;^{>}x = \;^{>}O_n$, т.е. $\det(A) \neq 0$ и $\exists A^{-1}$. Кроме того, отсюда следует неравенство: 
	\begin{equation}
	\llabel{2_4}
	\lVert \;^{>}x \rVert \geqslant \frac{1}{diag(A)} \cdot \lVert \;^{>}b \rVert
	\end{equation}
	
	С другой стороны $\;^{>}x = A^{-1} \;^{>}b$.
	
	Если $\lVert \;^{>}b \rVert = 1$, то из \lref{2_4} следует $\lVert \;^{>}x \rVert \leqslant \frac{1}{diag(A)} \Rightarrow \lVert A^{-1} \rVert \leqslant \frac{1}{diag(A)}$, т.к. $\lVert A^{-1} \rVert = max \lbrace \lVert A^{-1} \;^{>}b \rVert : \lVert \;^{>}b \rVert = 1 \rbrace$
\end{proof}

\begin{definition}
	Матрица $A \in L (\mathbb{R},n)$ называют трёхдиагональной, если $A = \left(
	\begin{matrix}
	b_1 & c_1 & 0 & 0 & ... & 0 \\
	a_2 & b_2 & c_2 & 0 &... & 0 \\
	0 & a_3 & b_3 & c_3 & ... & 0 \\
	... & ... & ... & ... & ... & ...\\
	0 & ... & 0 & a_{n-1} & b_{n-1} & c_{n-1} \\
	0 & ... & 0 & 0 & a_{n} & b_{n}
	\end{matrix}
	\right)$
\end{definition}

\textbf{Метод прогонки}

Рассмотрим СЛАУ \lref{2_1}, где матрица $A \in L(\mathbb{R},n)$ трёхдиагональная матрица имеет диагональное преобладание, т.е. $|b_i| >|a_i| + |c_i|, i \in \mathbb{N}$. Тогда для решения такой СЛАУ используют метод прогонки, состоящий из прямого и обратного хода.

\textit{A) Прямой ход}
\begin{enumerate}
	\item 
	Из первого уравнения $b_1 x^1 + c_1 x^2 =y^1 \Rightarrow x^1 = \frac{y^1 - c_2 x^2}{b^1} \Rightarrow x^1 = L_1 x^2 + M_1$, где $L_1 = - \frac{c_1}{b_1}$ и $M_1 = \frac{y^1}{b_1}$
	
	\item 
	Выражение $x^1 = L_1 x^2 + M_1$ подставим во второе уравнение слау и выражаем $x^2 = L_2 x^3 + M_2$
	
	\item
	Выражение $x^2 = L_2 x^3 + M_2$ подставляем в третье уравнение и получаем $x^3 = L_3 x^4 + M_3$
	
	...
	
	\item[n-1.]
	Получаем $x^{n-1} = L_{n-1} x^n + M_{n-1}$
	
	\item[n.]
	Подставляем $x^{n-1} = L_{n-1} x^n + M_{n-1}$ в n-е уравнение и $x^n = M_n$
\end{enumerate} 

\textit{Б) Обратный ход при известном $x^n = M_n$}
\begin{enumerate}
	\item[n-1.]
	Из (n-1) находим $x^{n-1} = L_{n-1} x^n + M_{n-1}$
	
	\item[n-2.]
	Из (n-2) находим $x^{n-2} = L_{n-2} x^{n-1} + M_{n-2}$
	
	...
	
	\item[1.]
	Из (1) находим $x^{1} = L_{1} x^{2} + M_{1}$
\end{enumerate}
\end{document}