\documentclass[__main__.tex]{subfiles}

\begin{document}

\qtitle{00}
Обозначения.

\textbf{Векторы и матрицы}

\begin{itemize}
	\item 
	$\;^{>}\mathbb{R} = \lbrace \;^{>}a = 
	\left(
	\begin{matrix}
	a_1 \\
	a_2 \\
	... \\
	a_n
	\end{matrix} 
	\right)
	= [a^1,a^2, ... ,a^n> : a^1, a^2, ... , a^n \in \mathbb{R}
	\rbrace$ - линейное пространство $\mathbb{R}$ - столбцов высоты $n$.
	
	\item 
	$\;^{<}\mathbb{R} = \lbrace \;^{<}a = < a_1,a_2, ... , a_n ] : a_1, a_2, ... , a_n \in \mathbb{R}
	\rbrace$ - линейное пространство $\mathbb{R}$ - строк длины $n$.
	
	\item
	$\mathbb{R}^n = \lbrace (a_1, a_2, ... , a_n): a_1, a_2, ... , a_n \in \mathbb{R} \rbrace$ - совокупность $\mathbb{R}$ - кортежей размера $n$
	
	\item
	$L(\mathbb{R}, n, m) = \lbrace A = (a^i_j)^m_n = \left( 
	\begin{matrix}
	a^1_1 & a^1_2 & ... & a^1_n \\
	... & ... & ... & ... \\
	... & ... & ... & ... \\
	a^m_1 & a^m_2 & ... & a^m_n
	\end{matrix}
	\right): \forall i \in \mathbb{N}_m, \forall j \in \mathbb{N}_n \rightarrow a_j^i \in \mathbb{R} \rbrace$ - линейное пространство $\mathbb{R}$-матриц размера $m \times n$ ($m$ строк, $n$ столбцов).
	
	\item
	$L (\mathbb{R},n,n) = L (\mathbb{R},n)$ - линейное пространство квадратных матриц.
	
	\item
	$OL(\mathbb{R},n) = \lbrace Q \in L(\mathbb{R},n): ^{T}Q \cdot Q = E_n = (\delta ^i_j)^n_n \rbrace$ - группа ортогональных матриц размера $n \times n$.
	
	\item
	$GL(\mathbb{R},n) = \lbrace A \in L(\mathbb{R},n): \det (A) \neq 0 \rbrace$ - группа невырожденных матриц размера $n \times n$.
	
	\item
	$A = (a^i_j)^m_n = < \;^{>}a_1, \;^{>}a_2, ... , \;^{>}a_n] = [\;^{<}a^1, \;^{<}a^2, ... , \;^{<}a_m >$, где $\;^{>}a_j = [a^1_j, a^2_j, ... , a^m_j >$ для $j = \overline{1,n}$, $\;^{<}a^i = < a^i_1, a^i_2, ... , a^i_n]$ для $i = \overline{1,m}$
\end{itemize}

\textbf{Стандартные базисы}

\begin{itemize}
	\item 
	$\;^{>}\varepsilon_n = (\;^{>}e_1, \;^{>}e_2, ... , \;^{>}e_n)$ - стандартный базис $\;^{>}\mathbb{R}^n$, $\;^{>}e_j = (\delta^i_j)^n$ для $j \in \mathbb{N}_n$.
	
	\item 
	$\;^{<}\varepsilon_n = (\;^{<}e^1, \;^{<}e^2, ... , \;^{<}e^n)$ - стандартный базис $\;^{<}\mathbb{R}^n$, $\;^{<}e^i = (\delta^i_j)_n$ для $i \in \mathbb{N}_n$.
\end{itemize}

\textbf{Умножение векторов и матриц}

$A = < \;^{>}a_1, \;^{>}a_2, ... , \;^{>}a_n ] = (a^i_j)^m_n$

$B = [ \;^{<}b^1, \;^{<}b^2, ... , \;^{<}b^k > = (b^p_q)^k_m$

$B \cdot A = (\;^{<}b^p \cdot \;^{>}a_j)^k_n$

Если $\;^{>}a \in \;^{>}\mathbb{R}^n$ и $\;^{<}b \in \;^{<}\mathbb{R}^n$, то $^{T}(\;^{>}a) = \;^{<}a$ и $\;^{<}b \cdot \;^{>}a = \sum_{i = 1}^{n} b_i \cdot a^i$

\textbf{Гомоморфизмы}

Пусть $X$ и $Y$ - линейные пространства над полем $\mathbb{R}$. Тогда $Hom(X,Y)$ - линейное пространство гомоморфизмов их $X$ в $Y$.

Пусть $Z$ - линейное пространство над $\mathbb{R}$, $\hat{A} \in Hom(X,Y)$ и $\hat{B} \in Hom(Y,Z) \rightarrow \hat{B} \circ \hat{A} \in Hom (X,Z)$, где $\hat{B} \circ \hat{A} (x) = \hat{B}(\hat{A}(x))$ для $\forall x \in X$ ($\circ$ - знак суперпозиции).

\textbf{Гомоморфизмы в нормированных пространствах}

Пусть $X \in (X, \lVert \cdot \rVert_X)$ и $Y \in (Y, \lVert \cdot \rVert_Y)$ - два нормированных пространства и $\hat{F} \in Hom(X,Y)$. $Hom_c(X,Y)$ - линейное пространство непрерывных гомоморфизмов из $X$ в $Y$.

\begin{statement}
	Гомоморфизм непрерывен ($\hat{F} \in Hom_c(X,Y)$) $\Leftrightarrow \hat{F}$ ограничен, т.е. ($\forall x \in X \Rightarrow \lVert \hat{F}(x) \rVert_Y \leqslant c \lVert x \rVert_X$, где $c \geqslant 0$).
\end{statement}

\begin{definition}[Подчинённая норма]
	Если $\hat{F} \in Hom_c(X,Y)$, то подчинённая норма $\lVert \hat{F} \rVert$ имеет вид: $\lVert \hat{F} \rVert = sup \lbrace \lVert \hat{F}(x) \rVert_Y : x \in X и \lVert x \rVert_X = 1 \rbrace$.
\end{definition}

\begin{statement}
	$Hom_c(X,Y) = (Hom_c(X,Y), \lVert \cdot \rVert)$ - нормированное пространство, где $\lVert \cdot \rVert$ - подчинённая норма. Если  $Y = (Y, \lVert \cdot \rVert_Y)$ - банахово пространство (любая фундаментальная последовательность сходится), то $Hom_c(X,Y)$ - банахово пространство.
\end{statement}

\begin{statement}
	$Hom_c(X, \mathbb{R}) = X*$ - банахово, т.к. $\mathbb{R} = (\mathbb{R}, | \cdot |)$ - банахово.
\end{statement}

\begin{statement}
	Если $A = (a^i_j)^m_n$, то $\lVert A \rVert = max \lbrace \sum_{j=1}^{n}| a^i_j | : i=\overline{1,m} \rbrace = \lVert A \rVert_r$
\end{statement}

Пусть $A \in L(\mathbb{R},n,m)$, $\hat{A} \in Hom(\;^{>}\mathbb{E}^n, \;^{>}E^m)$ - табличный гомоморфизм. Тогда $\lVert A \rVert_e  = \lVert \hat{A} \rVert = max {\lVert A \;^{>}x \rVert : \lVert \;^{>}x \rVert = 1}$

$\lVert A \;^{>}x \rVert^2 = \;^{T}(A \;^{>}x) \cdot A \cdot \;^{>}x = \;^{>}x \cdot \;^{T}A \cdot A \cdot \;^{>}x$

\begin{definition}
	a) $Spr(A) = \lbrace \lambda_1, \lambda_2, ... , \lambda_n \rbrace$ - совокупность собственных значений А. $\rho_{Spr}(A) = max \lbrace |\lambda_1|, |\lambda_2|, ... , |\lambda_n| \rbrace$ - спектральный радиус матрицы А.
	
	б) Если $B \in L(\mathbb{R},n)$ - симметрическая матрица, то $Spr(B) \subset \mathbb{R}$ и $B \geqslant 0 (\;^{<}x \cdot B \cdot \;^{>}x \geqslant 0$ для $\forall \;^{>}x \in \;^{>} \mathbb{R})$, то $Spr(B) \subset \mathbb{R}_+ = \lbrace r \in \mathbb{R}, r \geqslant 0 \rbrace$. Кроме того, $B = Q \cdot \left(
	\begin{matrix}
	\lambda_1 & 0 & ... & 0 \\
	0 & \lambda_2 & ... & 0 \\
	... & ... & ... & ... \\
	0 & 0 & ... & \lambda_n
	\end{matrix}
	\right) \cdot \;^{T}Q$, где $Q = < \;^{>}q_1, \;^{>}q_2, ... , \;^{>}q_n] \in OL(\mathbb{R},n)$
	$B \cdot \;^{>}q_i = \lambda_i \;^{>}q_i$ для $i=\overline{1,n}$.
\end{definition}

\end{document}