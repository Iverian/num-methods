\documentclass[__main__.tex]{subfiles}
\begin{document}

\qtitle{30}
Пространство сплайнов нулевой степени единичного дефекта, его стандартный базис.\\

\begin{definition}
    (Интерполяционного сплайна 0-й степени)\\
    Интерполяционным сплайном $spl_0\left(B,\;^>y\right)$ для $B$ - сеточной функции $^{>}y = [y_1,y_2,\cdots, y_k>\; \in \\ ^{>}R^{|B|}(B)$ называют определенную на $[a,b]$ функцию $\varphi = spl_0\left(B,^{>}y\right)$, для которой $\varphi(\tau) = y_i$, если $\tau \in s_i$ для $i = \overline{1,k}$
\end{definition}
Таким образом $spl_0\left(B,^>y\right)$ - кусочно постоянная функция.\\ Для стандартного базиса $(^>e_1,^>e_2,\cdots,^>e_{|B|})$ пространства $^>R^{|B|}$, где $^>e_i = [0,0,\cdots,1,\cdots,0,0]$, вводится базис $H = (h_1,h_2,\cdots, h_k)$ нормированного пространства интерполяционных сплайнов нулевой степени $spl_0(B)$ на сетке $B$, где $h_i = spl_0\left(B,^>e_i\right)$ для $i=\overline{1,k}$.\\
Следовательно сплайн
\begin{gather*}
    spl_0\left(B,^>y\right) = \sum_{i=1}^{n}y_i spl_0\left(B,^>e_i\right)
\end{gather*}
является решением задачи интерполяции для $B$ - сеточной функции
\begin{gather*}
    ^>y = [y_1,\cdots,y_k> \in\;^>R^{|B|}(B)
\end{gather*}

\end{document}