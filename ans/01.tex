\documentclass[__main__.tex]{subfiles}

\begin{document}

\qtitle{01}
Приближённое описание чисел, абсолютная и относительная погрешности. Арифметика вычислений с заданными погрешностями. Метод Жордана-Гаусса с выбором ведущего элемента.

Число $a^{*} \in \mathbb{R}$ задаётся в виде $(a,\Delta a)$, где $a \in \mathbb{R}$ - приближённое значение $a^{*}$ и $\Delta a \in \mathbb{R}_{(+)} = \lbrace r \in \mathbb{R}: r>0 \rbrace$ - \textbf{абсолютная погрешность приближения}, т.е. $a^{*} \in (a - \Delta a; a + \Delta a)$.

Если $0 \notin (a - \Delta a; a + \Delta a)$, то говорят, что число $(a,\Delta a)$ - отделено от нуля и используют обозначение $(a,\Delta a) \neq 0$.

Если $(a,\Delta a) \neq 0$, т.е. $|a| > \Delta a$, то число $\frac{\Delta a}{|a|}$ - называют \textbf{относительной погрешностью приближения \textit{a}}.

\textbf{Арифметические операции}
\begin{itemize}
	\item 
	$(a,\Delta a) \pm (b,\Delta b) = (a \pm b,\Delta a \pm \Delta b) $ 
	
	\item 
	$(a,\Delta a) \cdot (b,\Delta b) = (a \cdot b,|a| \Delta b + |b| \Delta a) $
	
	\item 
	Если $(a,\Delta a) \neq 0$, то $\frac{1}{(a,\Delta a)} = \left(\frac{1}{a}, \frac{\Delta a}{a^2} \right)$. Например $(a,\Delta a) = (10^{-6},10^{-7}) \rightarrow \frac{1}{(a,\Delta a)} = (10^6,10^5) \rightarrow \text{опасная операция}$
	
	\item
	Если $(a,\Delta a) \neq 0$, то $\frac{(b, \Delta b)}{(a, \Delta a)} = (b,\Delta b) \cdot \left( \frac{1}{a}, \frac{\Delta a}{a^2} \right) = \left( \frac{b}{a}, |b| \frac{\Delta a}{a^2} + \frac{1}{|a|} \Delta b \right)$
\end{itemize}

\textbf{Метод Жордана-Гаусса с выбором ведущего элемента}

Рассмотрим совместную СЛАУ
\begin{equation}
\llabel{1_1}
A \cdot \;^{>}x = \;^{>}b.
\end{equation}
где $\;^{>}x \in \;^{>}\mathbb{R}^n$ - решение СЛАУ \lref{1_1}. Сначала рассмотрим \textbf{схему единственного деления}.

\textit{Прямой ход}

\begin{enumerate}
	\item 
	Целью этого шага является исключение неизвестного $x_1$ из уравнений с номерами $i=\overline{2,m}$. Предположим, что коэффициент $a_{1 1} \neq 0$. Будем называть его \textit{главным элементом 1-го шага}. Найдём величины $\mu_{i 1} = \frac{a_{i 1}}{a_{1 1}}, i = \overline{2,m}$, называемое \textit{множителями 1-го шага}. Вычтем последовательно из второго, третьего, ..., $m$-го уравнений системы \lref{1_1} первое уравнение, умноженное соответственно на $\mu_{2 1}, \mu_{3 1}, ..., \mu_{m 1}$. Это позволит обратить в нуль коэффициенты при $x_1$ во всех уравенинях, кроме первого. В результате получим эквивалентную систему 
	
	\begin{gather}
		\llabel{1_2}
		\begin{cases}
			a_{1 1} x_1 + a_{1 2} x_2 + ... + a_{1 m} x_m = b_1 \\
			a_{2 2}^{(1)} x_2 + ... + a_{2 m}^{(1)} x_m = b_2^{(1)} \\
			... \\
			a_{m 2}^{(1)} x_2 + ... + a_{m m}^{(1)} x_m = b_m^{(1)}
		\end{cases}
	\end{gather}
	в которой $a^{(1)}_{i j}$ и $b^{(1)}_i$ ($i, j = \overline{2,m}$) вычисляются по формулам
	
	\begin{equation}
	\llabel{1_3}
	a^{(1)}_{i j} = a_{i j} - \mu_{i 1} a_{1 j}, b_i^{(1)} = b_i - \mu_{i 1} b_1
	\end{equation}
	
	\item 
	Целью этого шага является исключение неизвестного $x_2$ из уравнений с номерами $i = \overline{3,m}$. Пусть $a^{(1)}_{2 2} \neq 0$, где $a^{(1)}_{2 2}$ - коэффициент, называемый \textit{главным элементом 2-го шага}. Вычислим множители 2-го шага $\mu_{i 2} = \frac{a^{(1)}_{i 2}}{a^{(1)}_{2 2}}, i = \overline{3,m}$ и вычтем последовательно из третьего, четвёртого, ..., $m$-го уравнений системы \lref{1_2} второе уравнение, умноженное соответственно на $\mu_{3 2}, \mu_{4 2}, ..., \mu_{m 2}$. В результате получим систему
	
	\begin{gather}
		\llabel{1 4}
		\begin{cases}
			a_{1 1} x_1 + a_{1 2} x_2 + ... + a_{1 m} x_m = b_1 \\
			a_{2 2}^{(1)} x_2 + ... + a_{2 m}^{(1)} x_m = b_2^{(1)} \\
			a^{(2)}_{3 3} x_3 + ... + a^{(2)}_{3 m} x_m = b_3^{(2)} \\
			... \\
			a^{(2)}_{m 3} x_3 + ... + a^{(2)}_{m m} x_m = b_m^{(2)}
		\end{cases}	
	\end{gather}
	
	Здесь коэффициенты $a^{(2)}_{i j}$ и $b^{(2)}$ ($i, j = \overline{3,m}$) вычисляются по формулам
	
	\begin{gather*}
		\llabel{1_5}
		a^{(2)}_{i j} = a_{i j}^{(1)} - \mu_{i 2} a_{2 j}^{(1)}, b_i^{(2)} = b_i^{(1)} - \mu_{i 2} b_2^{(1)}
	\end{gather*}
	
	\item[k.]
	В предположении, что \textit{главный элемент} $k$-го шага $a^{(k-1)}_{k k}$ отличен от нуля, вычислим \textit{множители k-го шага} $\mu_{i k} = \frac{a_{i k}^{(k-1)}}{a_{k k}^{(k-1)}}, i = \overline{k+1,m}$ и вычтем последовательно из $(k+1)$-го, ..., $m$-го уравнений полученной на предыдущем шаге системы $k$-е уравнение, умноженное соответственно на $\mu_{k+1, k}, \mu_{k+2, k}, ..., \mu_{m k}$.
	
	После $(m-1)$-го шага исключения получим систему уравнений 
	
	\begin{gather}
		\llabel{1_6}
		\begin{cases}
			a_{1 1} x_1 + a_{1 2} x_2 + ... + a_{1 m} x_m = b_1 \\
			a_{2 2}^{(1)} x_2 + ... + a_{2 m}^{(1)} x_m = b_2^{(1)} \\
			a^{(2)}_{3 3} x_3 + ... + a^{(2)}_{3 m} x_m = b_3^{(2)} \\
			... \\
			a^{(m-1)}_{m m} x_m = b_m^{(m-1)}
		\end{cases}
	\end{gather}
	
	матрица $A^{(m-1)}$ которой является верхней треугольной. На этом вычисления прямого хода заканчиваются
\end{enumerate}

\textit{Обратный ход}

Из последнего уравнения системы \lref{1_6} находим $x_m$. Подставляя найденное значение $x_m$ в предпоследнее уравнение, получим $x_{m-1}$. Осуществляя обратную подстановку, далее последовательно находим $x_{m-2}, x_{m-3}, ..., x_1$. Вычисления неизвестных здесь проводятся по формулам

$$x_m = \frac{b_m^{(m-1)}}{a^{(m-1)}_{m m}}$$

$$x_k = \frac{b_k^{(k-1)} - a^{(k-1)}_{k, k + 1} x_{k + 1} - ... - a_{k m}^{(k-1)} x_m}{a_{k k}^{(k-1)}}, k = \overline{m-1, 1}$$

Вернёмся к \textbf{методу выбора главного элемента по столбцу.}

На $k$-м шаге прямого хода коэффициенты уравнений системы с номерами $i = \overline{k+1,m}$ преобразуются по формулам

$$a^{(k)}_{i j} = a^{(k-1)}_{i j} - \mu_{i k} a^{(k-1)}_{k j}$$

$$b_i^{(k)} = b_i^{(k-1)} - \mu_{i k} b_k^{(k-1)}, i = \overline{k+1,m}.$$

Интуитивно ясно, что во избежание сильного роста коэффициентов системы и связанных с этим ошибок нельзя допускать появления больших множителей $\mu_{i k}$.

В методе Жордана-Гаусса с выбором главного элемента по столбцу гарантируется, что $|\mu_{i k}| \leqslant 1$ для всех $k = \overline{1, m-1}$ и $i = \overline{k+1,m}$. Отличие этого варианта метода Жордана-Гаусса от схемы единственного деления заключается в том, что на $k$-м шаге исключения в качестве главного элемента выбирают максимальный по модулю коэффициент $a_{i_k k}$ при неизвестной $x_k$ в уравнениях с номерами $i = \overline{k,m}$. Затем соответствующее выбранному коэффициенту уравнение с номером $i_k$ меняют местами с $k$-м уравнением системы для того, чтобы главный элемент занял место коэффициента $a_{kk}^{(k-1)}$.

После этой перестановки исключение неизвестного $x_k$ производят, как в схеме единственного деления.

Вообще, ещё существует \textbf{метод Жордана-Гаусса с выбором главного элемента по всей матрице}.

В этой схеме допускается нарушение естественного порядка исключения неизвестных.

На первом шаге метода среди элементов $a_{i j}$ определяют максимальный по модулю элемент $a_{i_1 j_1}$. Первое уравнение системы и уравнение с номером $i_1$ меняют местами. Далее стандартным образом производят исключение неизвестного $x_{j_1}$ из всех уравнений, кроме первого.

На $k$-м шаге метода среди коэффициентов $a_{ij}^{(k-1)}$ при неизвестных в уравнениях системы с номерами $i = \overline{k,m}$ выбирают максимальный по модулю коэффициент $a_{i_k j_k}^{(k-1)}$. Затем $k$-е уравнение и уравнение, содержащее найденный коэффициент, меняют местами и исключают неизвестное $x_{j_k}$ из уравнений с номерами $i = \overline{k+1,m}$.

На этапе обратного хода неизвестные вычисляют в следующем порядке: $x_{j_m}, x_{j_{m-1}}, ..., x_{j_1}$.
\end{document}