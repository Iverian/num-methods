\documentclass[__main__.tex]{subfiles}

\begin{document}

\qtitle{09}
Многомерная модель линейной регрессии.\\

Эта модель имеет вид:
\begin{gather}
    \llabel{9:1}
    y = \alpha_0+\alpha_1x_1 + \alpha_2x_2 + ... + \alpha_kx_k + \epsilon
\end{gather}
где $^>\alpha = [\alpha_0,\alpha_1...\alpha_k>$ - вектор неизвестных параметров тренда $\widetilde{y} = \alpha_0 + \alpha_1x_1+...+\alpha_kx_k$ модели \lref{9:1} и $\epsilon \sim N(0,\sigma)$ - случайная составленная модель \lref{9:1}.

\begin{gather}
    ^<x^1 = <x_1^1,x_2^1,...,x_k^1]\\
    ^<x^2 = <x_1^2,x_2^2,...,x_k^2]\\
    ^<x^n = <x_1^n,x_2^n,...,x_k^n]
\end{gather}

В эксперименте для $n>k+1$ значений факторов $x_1...x_k$ модели \lref{9:1} измеряют соответствующие значения $y_1...y_n$ модели \lref{9:1}. Эти данные представляют в виде таблицы

\begin{gather}
    \llabel{9:2}
    \begin{array}{crrrr}
        i & x_1   & x_2   & ... & x_ky     \\
        1 & x_1^1 & x_2^1 & ... & x_k^1y_1 \\
        2 & x_1^2 & x_2^2 & ... & x_k^2y_2 \\
        ...                                \\
        n & x_1^n & x_2^n & ... & x_k^ny_n \\
    \end{array}
\end{gather}

Из модели \lref{9:1} и таблицы \lref{9:2} получают для оценки $^>\alpha$ приближенную СЛАУ:

\begin{gather}
    \llabel{9:6}
    \begin{cases}
        \alpha_0+\alpha_1x_1^1+...+\alpha_kx_k^k = y_1 \\
        ...                                            \\
        \alpha_0+\alpha_1x_1^n+...+\alpha_kx_k^n = y_n \\
    \end{cases}
    \Longleftrightarrow X\cdot ^>\alpha = ^>y\\
\end{gather}

\begin{gather}
    \llabel{9:3}
    X =
    \begin{pmatrix}
        1 & x_1^1 & ... & x_k^1 \\
        1 & x_1^2 & ... & x_k^2 \\
        ...                     \\
        1 & x_1^n & ... & x_k^n
    \end{pmatrix}
    \qquad ^>y =
    \begin{pmatrix}
        y_1 \\
        ... \\
        y_n
    \end{pmatrix}
\end{gather}

В качестве оценки $^>\hat{\alpha}$ СЛАУ \lref{9:3} выбирается решение задачи:

\begin{gather}
    \llabel{9:4}
    y(^>\alpha) = || X ^>\alpha - ^>\alpha||^2_e \to min
\end{gather}
где
\begin{flalign*}
    ||X {^>\alpha} - {^>y}||^2
    =&
    \left< X {^>\alpha} - {^>y}; X{^>\alpha} - {^>y}\right>_e
    =
    {^T}(X {^>\alpha} - {^>y})(X{^>\alpha} - {^>y})
    =\\
    =&
    ({^>\alpha}{^TX} - {^<y})(X{^>\alpha}-{^>y})
    =\\
    =&
    {^<\alpha}{^TX}X{^>\alpha}{^>\alpha}{^TX}{^>y} - ^>y X ^>\alpha + ||^>y||^2_e
    =\\
    =&
    ^>\alpha ^TX X ^>\alpha - 2^<\alpha ^+X ^>y + ||^>y||^2_e
\end{flalign*}
Отсюда для решения задачи \lref{9:4} получают СЛАУ:
\begin{gather}
    \begin{cases}
        \llabel{9:5}
        \frac{\partial y}{\partial \alpha_0} = 0 \\
        ...                                      \\
        \frac{\partial y}{\partial \alpha_k} = 0
    \end{cases}
    \Longleftrightarrow 2 ^TX X ^>\alpha - 2 ^TX ^>y = ^>0_k \Longleftrightarrow ^TX X ^>\alpha = ^TX ^>y
\end{gather}

СЛАУ \lref{9:5} называется нормалью для СЛАУ \lref{9:6}.\\
Если $rang(X) = k , n>k+1$, то $^TX X \in GL(R,k)$.\\
В качестве оценки решения СЛАУ \lref{9:10} выбирается МНК- решение СЛАУ \lref{9:10}, то есть решение СЛАУ \lref{9:5}:\\
\begin{gather}
    \llabel{9:7}
    \begin{matrix}
        \hat{\alpha_0} \\
        ...            \\
        \hat{\alpha_k}
    \end{matrix}
    = (^TXX)^{-1}\cdot ^TX^>y
\end{gather}
Из вывода СЛАУ \lref{9:5} следует $g'' = \frac{\partial^2g}{\partial \alpha_i \partial \alpha_j} = 2 ^TXX>0$
Следовательно, \lref{9:7} - решение задачи \lref{9:4}.\\
Для оценки $\hat{\sigma}^2$ величины $\sigma$ используется формула
\begin{gather}
    \hat{\sigma}^2 = \frac{||x ^>\hat{\alpha} - ^>y||^2_e}{n-k-1}
\end{gather}
\textbf{Пояснение за \lref{9:10}}
\begin{gather}
    \llabel{9:10}
    \begin{cases}
        \alpha_0 + \alpha_1x_1 = y_1 \\
        \alpha_0 + \alpha_2x_2 = y_2 \\
        ...
        \alpha_0 + \alpha_1x_n = y_n
    \end{cases}
    \Longleftrightarrow X ^>\alpha = ^>y
\end{gather}
где

\begin{gather}
    X = \begin{matrix}
        1 & x_1 \\
        ...     \\
        1 & x_n
    \end{matrix}
    \qquad ^>y =
    \begin{matrix}
        y_1 \\
        ... \\
        y_n \\
    \end{matrix}
\end{gather}

Такая штука следует из модели парной линейной регрессии, где $^>\alpha = \left[\alpha_0,\alpha_1\right>$
\end{document}