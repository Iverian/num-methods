\documentclass[__main__.tex]{subfiles}

\begin{document}

\qtitle{40}
Сеточно-сплайновое аппроксимирование в пространстве непрерывных на отрезке функций. Аппроксимирование в гильбертовом пространстве.

Пример Сеточно-сплайнового аппроксимирования \\
Пусть $Y_0=\underline{M}_B([a;b],R),X_0=\underline{C}([a;b],R) \ ,A_{(.)}=(A_i)_N$ схема устранимо-разрывных сеток $[a;b]$\\
Кроме того, $\hat{H}_{(.)}=(\hat{A}_n)_N$ -- табулирование элементов пространства $X_0$ и $B_{(.)}=(\hat{\Psi}_n\in hom_c(^>R^{|A_k|},Spl_{|A_k|}(A_k))$ -- интерпретирование схемы сеточных пространств ($^>R^{|A_k|}(A_k))_N$), где $\hat{\Psi}_k(^>U)=\sum u_i Spl(A_{u_i}{^>C}_i)	$ для $^>u\in ^>R^{|A_k}(A_k)$  и  $k\in N$\\
Тогда $\forall k\in N$ определена аппроксимация $\hat{p}_k=\hat{\Psi}_k \circ \hat{A}_k \in hom_c(Y_0{^>R}^{|A_K|}$ элементов из $X_0$	где $||\hat{p}_k||=1$. Поэтому $\hat{p}_{(.)}=(\hat{p}_k)_N$ -- корректное аппроксимирование элементов из $X_0$  в пространстве $Y_0$

\end{document}