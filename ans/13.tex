\documentclass[__main__.tex]{subfiles}

\begin{document}

\qtitle{13}
Метод простой итерации для решения СЛАУ специального вида.


Рассмотрим СЛАУ:
\begin{gather*}
^>x=G\cdot \;^>x+\;^>f,
\end{gather*}
где $G=(g^i_j)^n_n \in L(\mathbb R, n)$, $^>f \in \;^>\mathbb R^n$ и $||G||<1$.\\
Для решения этой СЛАУ используем метод простой итерации $\operatorname{itr}(\varphi; \;^>x_0$ с началом $^>x_0\in \;^>\mathbb R^n$, где $\varphi(x)=G\cdot \;^>x+\;^>f$ для $^>x \in \;^> \mathbb R^n$. \\
Отображение $\varphi: \;^>\mathbb R \rightarrow \;^>\mathbb R$ сжимающее, так как 
\begin{gather*}
||\varphi(\;^>y)-\varphi(\;^>x)||=||G\cdot \;^>y -G\cdot \;^>x|| \leqslant ||G||\cdot ||\;^>y-\;^>x||, 
\end{gather*}
где $||G||<1$.\\
Пусть $^>x_*\in \;^>\mathbb R$ -- решение СЛАУ. Тогда используя итерационную последовательность $(x_k)_{\mathbb N}$ метода $\operatorname{\varphi, \;^>x_0}$ с рабочей формулой:
\begin{gather*}
^>x_k=G\cdot \;^>x_{k-1}+\;^>f, \qquad k\in \mathbb R
\end{gather*}
Получаем на $k$-ом шаге итерации оценку
\begin{gather*}
||\;^>x_k-\;^>x_*||\leqslant \frac{||G||^k}{1-||G||}||G\;^>x_0+\;^>f-\;^>x_0||\leqslant \frac{||G||^k}{1-||G||}(||G||-1)||\;^>x_0||+\frac{||G||^k}{1-||G||}||\;^>f||
\end{gather*}
Уточним эту оценку. Согласно рабочей формуле:
\begin{gather*}
\left\{
	\begin{gathered}
	^>x_1=G\cdot\;^>x_0+^>f; \hfill \\
	^>x_2=G\cdot\;^>x_1+^>f=G^2x_0+(E_n+G)\;^>f; \hfill \\
	^>x_3=G\cdot\;^>x_2+^>f=G^3x_0+(E_n+G+G^2)\;^>f; \hfill \\
	\cdots \\
	^>x_k=G^kx_0+(E_n+G+G^2+...+G^{k-1})\;^>f; \hfill \\
	\end{gathered}
\right.
\end{gather*}
\textbf{Замечание 1}\\
Согласно произведению Коши абсолютно сходяшихся рядов в банаховой алгебре $L(\mathbb R, n)$ получаем:
\begin{gather*}
(E_n-G)(E_n+G+G^2+...+G_k+...)=E_n \Longrightarrow (E_n-G)^{-1}=E_n+G+G^2+...+G^{k-1}+G^k+...
\end{gather*}
Из исходной СЛАУ следует:
\begin{gather*}
(E_n-G)\;^>x_*=\;^>f
\end{gather*}
То есть решение имеет вид:
\begin{gather*}
^>x_*=(E_n-G)^{-1}\cdot \;^>f =(E_n+G+G^2+...+G^{k-1})\cdot \;^>f+(G^k+G^{k+1}+...)\cdot \;^>f
\end{gather*}
Следовательно, получаем:
\begin{gather*}
||\;^>x_k-\;^>x_*||=||G^k\;^>x_0+G^k(E_n+G+G^2+...)\;^>f|| \leqslant ||G^k||\cdot ||\;^>x_0||+\frac{||G||^k}{1-||G||}||\;^>f||
\end{gather*}
Таким образом:
\begin{gather*}
||\;^>x_*-\;^>x_k||\leqslant ||G||^k\cdot ||\;^>x_0||+\frac{||G||^k}{1-||G||}||\;^>f||
\end{gather*}
\textbf{Замечание 2} (о СЛАУ, имеющей матрицу с диагональным преобладанием)\\

Рассмотрим СЛАУ:
\begin{gather*}
\left\{
	\begin{gathered}
	a^1_1x^1+a^1_2x^2+a^1_3x^3=b^1 \hfill \\
	a^2_1x^1+a^2_2x^2+a^2_3x^3=b^2 \hfill \\
	a^3_1x^1+a^3_2x^2+a^3_3x^3=b^3 \hfill \\
	\end{gathered}
\right. \Longleftrightarrow A\cdot \;^>x=\;^>b
\end{gather*}
где матрица $A=(a^i_j)^3_3$ имеет диагональное преобладание:
\begin{gather*}
|a^i_i|-\sum\limits^3_{j=1, j\neq i}|a^i_j|>0, \qquad i=\overline{1..3} 
\end{gather*}
Поделив $i$-ое уравнение этой СЛАУ на $a^i_i$ ($i=\overline{1..3}$), получаем равносильную СЛАУ:
\begin{gather*}
\left\{
	\begin{gathered}
	x^1=-\frac{a^1_2}{a^1_1}x^2-\frac{a^1_3}{a^1_1}x^3+\frac{b^1}{a^1_1} \hfill \\
	x^2=-\frac{a^2_1}{a^2_2}x^1-\frac{a^2_3}{a^2_2}x^3+\frac{b^2}{a^2_2} \hfill \\
	x^3=-\frac{a^3_1}{a^3_3}x^1-\frac{a^3_2}{a^3_3}x^2+\frac{b^3}{a^3_3} \hfill \\
	\end{gathered}
\right. \Longleftrightarrow  \;^>x=G\cdot \;^>x+\;^>f
\end{gather*}
где $^>f=\left[\frac{b^1}{a^1_1}; \frac{b^2}{a^2_2}; \frac{b^3}{a^3_3}\right\rangle$, $G=\left(
\begin{matrix}
0 & -\frac{a^1_2}{a^1_1} & -\frac{a^1_3}{a^1_1} \\
-\frac{a^2_1}{a^2_2} & 0 & -\frac{a^2_3}{a^2_2} \\
-\frac{a^3_1}{a^3_3} & -\frac{a^3_2}{a^3_3} & 0 \\
\end{matrix}
\right)$ и $||G||<1$, т.к. $\frac{\sum\limits^3_{j=1, j\neq i}|a^i_j|}{|a^i_i|}<1$ для $i=\overline{1..3}$.\\
Полученная СЛАУ решается методом простой итерации.\\
\end{document}