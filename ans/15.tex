\documentclass[__main__.tex]{subfiles}

\begin{document}

\qtitle{15}
Метод Зейделя как модификация метода простой итерации для решения СЛАУ.


Для СЛАУ вида 
\begin{gather*}
^>x=G\cdot \;^>x+\;^>f
\end{gather*}
можно использовать итерационный метод Зейделя с началом в $^>x_0\in \;^>\mathbb R^n$ и рабочей формулой 
\begin{gather*}
^>x_k=P\cdot \;^>x_{k-1}+Q\cdot \;^>x_k+ \;^>f,
\end{gather*}
где $P=\left(
\begin{matrix}
g^1_1 & g^1_2 & \cdots & g^1_n \\
0 & g^2_2 & \cdots & g^2_n \\
\vdots & \cdots & \ddots & \vdots \\
0 & \cdots & \cdots & g^n_n \\
\end{matrix}
\right)$, $Q = \left(
\begin{matrix}
0 & \cdots & \cdots & 0 \\
g^2_1 & 0 & \cdots & 0 \\
\vdots & \cdots & \ddots & \vdots \\
g^n_1 & g^n_2 & \cdots & 0
\end{matrix}
\right)$.

Рассмотрим рабочую формулу для метода Зейделя для $n=3$:
\begin{gather*}
\left\{
	\begin{gathered}
	x^1_k=g^1_1x^1_{k-1}+g^1_2x^2_{k-1}+g^1_3x^3_{k-1} + f^1 \hfill \\
	x^2_k=g^2_1x^1_{k}+g^2_2x^2_{k-1}+g^2_3x^3_{k-1} + f^2 \hfill \\
	x^3_k=g^3_1x^1_{k}+g^3_2x^2_{k}+g^3_3x^3_{k-1} + f^3 \hfill \\
	\end{gathered}
\right.
\end{gather*}
\begin{theorem}
(о сходимости метода Зейделя) Если $||G||<1$, то метод Зейделя с рабочей формулой сходится к решению $^>x_*$ СЛАУ вида $^>x=G\cdot \;^>x+\;^>f$, то есть 
\begin{gather*}
\lim\limits_{k\rightarrow +\infty} \;^>x_k=\;^>x_*
\end{gather*}
\end{theorem}
\textbf{Замечание} (о сходимости метода Зейделя) \\
а) В условиях теоремы о сходимости метод Зейделя как правило сходится быстрее метода простой итерации. \\
б) Из рабочей формулы для $n=3$ (см. выше) следует, что
\begin{gather*}
(E_n-Q)\cdot \;^>x_k=P\cdot \;^>x_{k-1}+\;^>f \Longleftrightarrow \;^>x_k=(E_n-Q)^{-1}P\cdot \;^>x_k + (E_n-Q)^{-1}\cdot \;^>f \Longleftrightarrow \\
\Longleftrightarrow  \;^>x_k = \tilde{G}\cdot \;^>x_{k-1}+\;^>\tilde{f}
\end{gather*}
где $\tilde{G}=(E_n-Q)^{-1}\cdot P$ и $^>\tilde{f}=(E_n-Q)^{-1}\cdot \;^>f$.\\
Следовательно, метод Зейделя -- модифицированный метод простой итерации.
\end{document}