\documentclass[__main__.tex]{subfiles}
\begin{document}
	\qtitle{32}
	Пространство сплайнов второй степени единичного дефекта, его стандартный базис.
	\begin{definition}
		\label{32_1}
		(Интерполяционного сплайна 2-ой степени дефекта 1)\\
		Пусть $A = \langle a = \tau_0,\tau_1,\cdots, \tau_k = b\rangle$ - сетка отрезка $[a,b]$. Интерполяционным сплайном 2-ой степени дефекта 1 для $A$ - сеточной функции 
		\begin{gather*}
			^>y = \hat{A}(f) = [y_0, y_1, \cdots, y_k \rangle \in \; ^>R^{|A|}(A)
		\end{gather*}
		называется функция $\varphi \in C^1\left([a,b], R\right)$, для которой на отрезке $[\tau_{i-1}; \tau_i]$ для $\tau \in [\tau_{i-1}; \tau_i]$ справедливо равенство 
		\begin{gather*}
			\varphi(\tau) = a_i+b_i(\tau - \tau_i)+c_i(\tau - \tau_i)^2
		\end{gather*}
		где $a_i,b_i,c_i \in R$, причем
		\begin{enumerate}
			\item этот сплайн на отрезке $[\tau_0;\tau_1]$ линеен
			\item этот сплайн непрерывен и его значения в узлах сетки совпадают со значениями интерполируемой функции
			\item первая производная этого сплайна также непрерывна
		\end{enumerate}
	\end{definition}
	\begin{definition}
		(его стандартного базиса)\\
		Список $H = (h_1,h_2,\cdots, h_k)$ где $h_i = spl_2(A,\; ^>e_{i+1})$ для $i=\overline{0,k}$ является базисом нормированного пространства $spl_2(A)$
	\end{definition}
	Если сетка $A = \langle a=\tau_0, \tau_1,\cdots,\tau_k = b \rangle$ является равномерной и ее шаг $h = \frac{b-a}{k} = stp(A)$, то из определения (\ref{32_1}) для $i = \overline{1,k}$  на отрезке $\tau_{i-1};\tau_i$ находятся значения коэффициентов полинома $P_i(\tau) = a_i+b_i(\tau-\tau_i)+c_i(\tau-\tau_i)^2$:
	\begin{gather*}
		a_i = y_{i-1},\;\;\;\;\;\;\;b_i = b_{i-1}+2c_{i-1}h,\;\;\;\;\;\;\;c_i = \frac{y_i-y_{i-1}-b_ih}{h^2}
	\end{gather*}
	где $i=\overline{2,k}$
\end{document}