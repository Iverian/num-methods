\documentclass[__main__.tex]{subfiles}

\begin{document}

\qtitle{39}
Понятия базы аппроксимирования и аппроксимирования линейного многообразия в банаховом пространстве; понятия сходимости, аналитической корректности, устойчивости и корректности аппроксимирования. Пример.


Пусть $H_{(.)} = (H_k)_N - $ последовательность квази-базисов в пространстве $Y_0,$ что $\forall x_0\in X_0$ существует последовательность $y_{(.)}=<y_k\in [H_k]=Y_k)_N,$ сходящаяся к $X_0$ в пр-ве $Y_0$, то есть $x_0=lim_{k->\infty}y_k.$ Тогда последовательность $H_{(.)}$ называю базой аппроксимирования линейного многообразия $X_0$ \\

Пусть $H_{(.)}=(H_k \in Y_0)_N - $ база аппроксимирования $X_0$ в $Y_0$  и $\forall k \in N$ задана аппроксимация\\
$\hat{p}_k \in hom_c (Y_0,[H_k]=Y_k)$ элементов из $X_0$\\
Тогда последовательность $\hat{p}_{(.)}=(\hat{p}_k)_N $ называют аппроксимированием из элементов из $X_0$. Такое аппроксимирование  $\hat{p}_{(.)}$ называется сходящимся, если $\forall X_0 \exists lim_{R->\infty} \hat{p}_k (x_0) = Y_0$. Такое сходящееся аппроксимирование назовем аналитически-корректной , если $lim_{k->+\infty}\hat{p}_k(x_0)=x_0 \forall x_0\in X_0$\\
Так аналитически-корректное аппроксимирование $\hat{p}_{(.)}$ корректно, если оно устойчиво; то есть $(||\hat{p}_k||)_N$ - ограничена\\

\end{document}