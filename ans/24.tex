\documentclass[__main__.tex]{subfiles}

\begin{document}

\qtitle{24}
Понятие схемы функций Чебышёва, примеры таких традиционных схем. Задача интерполяции сеточной функции по системе функций Чебышёва.\\

Пусть на $[a, b]$ задан список $(\varphi_0, \varphi_1, \varphi_2, \text{...}, \varphi_k) = \Phi_k$ определенных и ограниченных на $[a, b]$ функций со значениями в $\mathbb{R}$

\begin{definition}
Список $\Phi_k$ называется системой функций пространства, если для любых чисел $\alpha^0, \alpha^1, \text{...}, \alpha^m in \mathbb{R}$ где $m \in \mathbb{Z}_{k+1}$ и $\alpha_m \neq 0$ ф-ии $g = \sum_{i = 0}^m d_i \varphi_i$ имеет на $[a, b]$ не более чем $m$ нулей. Кроме того, такие функции $g$ называются обобщенными полиномами степени $deg(g) = m$ индуцируемой системой $\Phi_k$.

Линейное пространство таких обобщенных полиномов будет обозначаться выражением $[\Phi_k] = [\varphi_0, \varphi_1, \text{...}, \varphi_k] = \{ \sum_{i = 0}^k \alpha^i \varphi_i, \; \alpha^i \in \mathbb{R}$ для $i = \overline{0, k} \}$ - линейная оболочка $\Phi_k$.

\end{definition}

\begin{definition}
Пусть $\varphi_{(\cdot)} = (\varphi_{n-1})_{\mathbb{N}}$ - последовательность определенных и ограниченных на $[a, b]$ функций со значениями в $\mathbb{R}$.\\
Последовательность $\varphi_{(\cdot)}$ называется схемой функций Чебышева на $[a, b]$ если для $\forall k \in \mathbb{Z}_+$ список $\Phi_k = (\varphi_0, \varphi_1, \text{...}, \varphi_k)$ является системой функций пространства на $[a, b]$.
\end{definition}

\textbf{Примеры систем функций Чебышева (одновременно это и схемы функций Чебышева на отрезке)}

\begin{itemize}

\item $\overline{\Phi}_k = (1, \tau, \tau^2, \text{...}, \tau^k)$ на $[a, b]$ - параболическая система

\item $\overline{\Phi}_k = (\Tau_0, \Tau_1, \text{...}, \Tau_k)$, где $T_i$ для $i = \overline{0, k}$ - полиномы Чебышева 1-го рода на $[-1, 1]$ - алгебраическая система 

\item $\overline{\Phi}_k = (e^{\beta_0 \tau}, e^{\beta_1 \tau}, \text{...}, e^{\beta_k \tau})$ - экспоненциальная система, где $\beta_0, \beta_1, \text{...}, \beta_k \in \mathbb{R}$ - попарно различные числа

\item $\overline{\Phi}_k = (1, \cos \tau, \cos 2 \tau, \text{...}, \cos(k \tau))$ - тригонометрическая система на $[0, \pi]$ 

\end{itemize}

\textbf{Задача интерполяции сеточной функции по системе функций Чебышёва.}

\begin{theorem}
Для функций ${}^{>}f \in {}^{>}\mathbb{R}^{|A|}(A)$ существует единственный обобщенный полином $g = x_0 h_0 + x_1 h_1 + \text{...} + x_k h_k$, где ${}^{>}x = [x_0, x_1, \text{...}, x_k > \in {}^{>}\mathbb{R}^{|A|}$, решающий задачу интерполяции функции ${}^{>} f$.
\end{theorem}

\begin{proof}
Рассмотрим задачу интерполяции по системе $(h_0, h_1, \text{...}, h_k)$ для нулевой $A$-сеточной функции ${}^{>}O_{|A|} = [0, 0, \text{...}, 0> \in {}^{>}\mathbb{R}^{|A|}(A)$ с помощью $h_{(\cdot)}$ - обобщенного полинома $G = y_0 h_0 + \text{...} + y_k h_k$, где вектор ${}^{>} y = [y_0, y_1, \text{...}, y_k> \in {}^{>}\mathbb{R}^{|A|}(A)$ определяется из СЛАУ:

\begin{gather}
\begin{cases}
y_0 h_0(r_0) + y_1 h_1 (r_0) + \text{...} + y_k h_k(r_0) = 0\\
y_0 h_0(r_1) + y_1 h_1 (r_1) + \text{...} + y_k h_k(r_1) = 0\\
\cdots\\
y_0 h_0(r_k) + y_1 h_1 (r_k) + \text{...} + y_k h_k(r_k) = 0\\
\end{cases}
\Leftrightarrow
{}^{>}H \cdot {}^{>} y = {}^{>} O_{|A|}
\label{answer_24_1}
\end{gather}

где $H = (h_j^i)_{|A|}^{|A|}$ - матрица СЛАУ \ref{answer_24_1} и $h_j^i = h_j(r_i)$ для $i, j \in \mathbb{Z}_{k + 1}$.

Тогда полином $G$ согласно СЛАУ \ref{answer_24_1}, является полиномом степени не выше $k$ и имеет на $[a; b]$ не менее $(k + 1)$ нулей. Но $(h_0, h_1, \text{...}, h_k)$ - система функций, следовательно, $G = 0$ на $[a; b]$, т.е $det(H) \neq 0$.\\

Поэтому СЛАУ $H \cdot {}^{>}x = {}^{>}f$ имеет единственное решение и задача интерполяции ${}^{>}f$ по системе функций Чебышева $(h_0, h_1, \text{...}, h_{k - 1}))$ имеет единственное решение.
\end{proof}

\end{document}